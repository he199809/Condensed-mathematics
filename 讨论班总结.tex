\documentclass[UTF8,12,a4paper]{ctexart}
%\documentclass[13pt]{extarticle}
\usepackage{amsmath,amsthm,amsfonts,amssymb,amscd}
\usepackage{latexsym}
\usepackage{amsmath}
\usepackage{amsthm}
\usepackage{setspace}   %设置行距的宏包
\usepackage{fancyhdr}
%\usepackage{geometry}   %设置页边距的宏包
\usepackage{titlesec}   %设置页眉页脚的宏包
\usepackage{tikz-cd}    %画交换图的宏包 
\usepackage{enumerate}
\usepackage{color}
\usepackage{amsfonts}
%\usepackage{extsizes}
\usepackage{tikz}
\usepackage{tcolorbox}
\usepackage[all]{xy}
\tcbuselibrary{skins, breakable, theorems}
\usepackage{enumerate}
\renewcommand{\baselinestretch}{1.4}

%\geometry{a4paper}%
%\geometry{left=3cm,right=2.5cm,top=2.5cm,bottom=2.5cm}  %设置 上、左、下、右 页边距
\pagestyle{plain}


%\newtcolorbox{mybox}{colback = red!25!white, colframe = red!75!black}
\definecolor{darkgreen}{rgb}{0,0.35,0}
\newtcolorbox{mybox}[2][]{width=10cm,colback = red!5!white, colframe = green!75!black, fonttitle = \bfseries,colbacktitle = red!55!yellow, enhanced,attach boxed title to top left={yshift=-2mm},	title=#2,#1}
%\newtcbtheorem{question}{}
%{enhanced, breakable,
%	colback = white, colframe = cyan, colbacktitle = cyan,
%	attach boxed title to top left = {yshift = -2mm, xshift = 5mm},
%	boxed title style = {sharp corners},
%	fonttitle = \sffamily\bfseries, separator sign = {).~}}{qst}



\theoremstyle{definition}

\newtheorem{thm}{{Theorem}\hspace{0.05pt}}[section]
\newtheorem{prop}[thm]{{Proposition}\hspace{0.05pt}}
\newtheorem{cor}[thm]{Corollary\hspace{0.05pt}}
\newtheorem{lem}[thm]{{Lemma}\hspace{0.05pt}}
\newtheorem{dfn}[thm]{{Definition}\hspace{0.05pt}}
\newtheorem{rem}[thm]{{Remark}\hspace{0.05pt}}
\newtheorem{exm}[thm]{{Example}\hspace{0.05pt}}



\newtheorem*{Q}{{Question}\hspace{0.05pt}}
\newtheorem*{A}{{Answer}\hspace{0.05pt}}
\newtheorem*{property}{{Property}\hspace{0.05pt}}



\newtheorem{conj}{Open Problem\hspace{0.05pt}}
\newtheorem*{prob}{Problem\hspace{0.05pt}}
\newtheorem*{fact}{{Fact}\hspace{0.05pt}}
\newtheorem{ex}{ \hspace{0.05pt}}[section]


\newtheorem*{cexm}{{Counterexample}\hspace{0.05pt}}

\newtheorem*{ntt}{{Notation}\hspace{0.05pt}}
%\newtheorem*{dfn*}{Definition\hspace{0.05pt}}
%\newtheorem{note}[thm]{Note}
\newtheorem*{pf}{Proof}
\newtheorem*{Solution}{Solution}
\usepackage[top=3.2cm,bottom=3.8cm,left=3.0cm,right=3.0cm]{geometry}
%\newcommand{\Spec}{\mathrm{Spec}\ }
%\newcommand{\Gal}{\mathrm{Gal}\ }
\newcommand{\q}{\mathfrak{q}}
\newcommand{\p}{\mathfrak{p}}
\newcommand{\m}{\mathfrak{m}}
%\newcommand{\a}{\mathfrak{a}}
\newcommand{\F}{\mathcal{F}}
\newcommand{\G}{\mathcal{G}}
%\newcommand{\H}{\mathcal{H}}
\newcommand{\Cond}{\text{Cond}}







\begin{document}
\title{\vspace{-6em}\Large{Condensed mathematics I}}
\date{}
\author{何力}
\maketitle	
\section{Condensed Sets}
\dfn 
Let $\mathcal{C}$ be a category. A Grothendieck topology on $\mathcal{C}$ consists of: for each object $X$ in $\mathcal{C}$, there is a collection $\text{Cov}(X)$ of sets $\{X_i\to X\}_{i\in I}$, satisfying the following three axioms:
\begin{itemize}
	\item [(i)] If $V\to X$ is an isomorphism, then $\{V\to X\}\in \text{Cov}(X).$
	\item [(ii)] If $\{X_i\to X\}_{i\in I}\in \text{Cov}(X)$ and $Y\to X$ is any arrow in $\mathcal{C}$, then the fiber products $X_i\times_{X} Y$ exist and $\{X_i\times_{X} Y\to Y\}_{i\in I}\in \text{Cov}(Y).$
	\item [(iii)] If $\{X_i\to X\}_{i\in I}\in \text{Cov}(X)$ and for each $i\in I$, $\{V_{ij}\to X_i\}_{j\in I_i}\in \text{Cov}(X_i)$, then $\{V_{ij}\to X\}_{i\in I, j\in I_i}\to \text{Cov}(X).$
\end{itemize}
We call elements of $\text{Cov}(X)$ coverings.

\dfn 
A site is a category $\mathcal{C}$ together with a Grothendieck topology.

\exm 
Let $\mathcal{C}=\text{ProFin}$, the category of all profinite sets. For $\{X_i\to Y\}_{i\in I}$ be a covering, we mean $I$ is a finite index and $\coprod_{i\in I} X_i\to Y$ is a surjection. We also call maps $\{X_i\to Y\}_{i\in I}$ finite jointly surjective families of maps.\\
Now, for the category $\text{ProFin}$ together its coverings, we call it the pro\'etale site of a point and denote it by $*_{\text{pro\'et}}$.

\dfn 
\begin{itemize}
	\item [(i)] For any site $\mathcal{C}$, we call a functor 
	$$\mathcal{F}:\mathcal{C}^\text{op}\rightarrow\text{Set}$$
	a presheaf of sets.
	\item [(ii)] For a presheaf of sets $\mathcal{F}:\mathcal{C}^\text{op}\rightarrow\text{Set}$, if for any $X\in\mathcal{C}$ and any covering $\{X_i\to X\}_{i\in I}\in \text{Cov}(X)$, we have
	$$
	\mathcal{F}(X)\stackrel{\sim}{\longrightarrow}\text{Eq}(\prod_{i\in I}\mathcal{F}(X_i)\rightrightarrows \prod_{i,j\in I}\mathcal{F}(X_i\times_{X} X_j)).
	$$
	Then we call $\mathcal{F}$ a sheaf of sets.
\end{itemize}

\dfn 
A condensed set $T$ is a sheaf of sets on $*_{\text{pro\'et}}$, i.e. a functor $T:*_{\text{pro\'et}}^{\text{op}}\to \text{Set}$ satisfying the sheaf condition.
\rem 
\begin{itemize}
	\item [(i)] Concretely, a condensed set $T$ is a functor $T:\text{ProFin}^\text{op}\rightarrow \text{Set}$, satisfying $T(\emptyset)=*$ and
	\begin{itemize}
		\item For any profinite sets $S_1, S_2$, the natural map
		$$
		T(S_1\sqcup S_2)\longrightarrow T(S_1)\times T(S_2)
		$$
		is a bijection.
		\item For any surjection $S^\prime\twoheadrightarrow S$ of profinite sets with fiber product $S^\prime\times_{S} S^\prime$ and two projections $p_1,p_2:S^\prime\times_{S} S^\prime\rightarrow S^\prime$, the map
		$$
		T(S)\stackrel{\sim}{\longrightarrow}\{x\in T(S^\prime)|\  p_1^*(x)=p_2^*(x)\in T(S^\prime\times_{S} S^\prime)\}
		$$
		is a bijection. In other words, $T$ maps the pullback diagram
		\begin{equation*}		
		\begin{tikzcd}
		S^\prime\times_{S} S^\prime \arrow[d, "p_1"'] \arrow[r, "p_2"] & S^\prime \arrow[d] \\
		S^\prime \arrow[r]                                             & S                 
		\end{tikzcd}
		\end{equation*}
		to a pullback diagram
		\begin{equation*}	
		\begin{tikzcd}
		T(S^\prime\times_{S} S^\prime) & T(S^\prime) \arrow[l, "p_2^*"'] \\
		T(S^\prime) \arrow[u, "p_1^*"] & T(S) \arrow[l] \arrow[u]       
		\end{tikzcd}
		\end{equation*}
	\end{itemize}
	\item [(ii)] The category $\text{ProFin}$ of all profinite sets is a large category.
\end{itemize}

\dfn 
$\kappa$ is an uncountable strong limit cardinal if $\kappa$ is uncountable and for any $\lambda<\kappa$, we have $2^\lambda<\kappa$.

\exm 
For any limit cardinal $\lambda$, i.e. if $\kappa<\lambda$, then $\kappa+1<\lambda.$ We define
$$\sqsupset_0=\aleph_0,\cdots, \sqsubset_{\alpha+1}=2^{\sqsubset_{\alpha}},$$
and let 
$$\sqsubset_{\lambda}=\bigcup_{\alpha<\lambda}\sqsubset_{\alpha},$$
then we can show that $\sqsubset_{\lambda}$ is an uncountable strong limit cartinal.

\ntt 
We let $\kappa\text{-ProFin}$ denote the category of all $\kappa$-small profinite sets, i.e. profinite sets whose cardinal less equal than $\kappa$. Let  $\text{Cond}_\kappa(\text{Set})=\text{Sh}(\kappa\text{-ProFin},\text{Set})$.

\rem 
If $\kappa^\prime>\kappa$ are two uncountable strong limit cardinals, and denote the inclusion by $i:\kappa\text{-ProFin}\hookrightarrow \kappa^\prime\text{-ProFin} $, then we have a forgetful functor
$$
\text{Cond}_{\kappa^\prime}(\text{Set})\longrightarrow \text{Cond}_\kappa(\text{Set});\ T\mapsto T\circ i.
$$
This forgetful functor admits a left adjoint $F:\text{Cond}_{\kappa}(\text{Set})\longrightarrow \text{Cond}_{\kappa^\prime}(\text{Set})$. $F$ is fully faithful and $F$ commutes with all colimits and all finite limits.\\
We define 
$$
\text{Cond}(\text{Set})=\bigcup_\kappa \text{Cond}_{\kappa}(\text{Set})=\underset{\kappa}{\underrightarrow{\text{lim}}}\  \text{Cond}_{\kappa}(\text{Set}).
$$

\exm 
Let $\text{Top}$ denote the category of all topological spaces. For each $T\in \text{Top}$, we can define $\underline{T}\in \text{Cond}(\text{Set})$ as follows:
$$
\underline{T}:\text{ProFin}^{\text{op}}\longrightarrow\text{Set};\ S\mapsto \underline{T}(S)=\text{Cont}(S,T)=\{\text{continuous maps from } S\text{ to } T\}.
$$
We need to check that $\underline{T}$ is a condensed set:
\begin{itemize}
	\item [(i)] $\underline{T}(S_1\sqcup S_2)=\text{Cont}(S_1\sqcup S_2, T)=\text{Cont}(S_1, T)\times \text{Cont}(S_2, T)=\underline{T}(S_1)\times\underline{T}(S_2).$
	\item [(ii)] For any surjection $g:S^\prime\twoheadrightarrow S$, let $p_1,p_2:S^\prime\times_{S} S^\prime\rightarrow S^\prime$ be the two projections. We need to show the following map is a bijection:
	$$
	\text{Cont}(S,T)\stackrel{\sim}{\longrightarrow}\{h:S^\prime\rightarrow T|\ hp_1=hp_2: S^\prime\times_{S} S^\prime\rightarrow T\};\  f\mapsto f\circ g. 
	$$
	Because $g$ is surjective, it is easy to show this map is an injection.\\
	Now, for any $h:S^\prime\rightarrow T$ with $hp_1=hp_2$, from the universal property of pushout(in our situation, the pullback square is also a pushout), we can find a unique $f$, s.t. the diagram commutes.
	\begin{equation*}	
	\begin{tikzcd}
	S^\prime\times_{S} S^\prime \arrow[r] \arrow[r] \arrow[r] \arrow[r, "p_2"] \arrow[d, "p_1"'] & S^\prime \arrow[d, "g"', two heads] \arrow[rdd, "h"] &   \\
	S^\prime \arrow[r, "g", two heads] \arrow[rrd, "h"']                                         & S \arrow[rd, "\exists!\ f", dashed]                   &   \\
	&                                                      & T
	\end{tikzcd}
	\end{equation*}
\end{itemize}

\dfn Let $X\in \text{Top}.$ The following are equivalent definition:
\begin{itemize}
	\item [(i)]$X\in \text{Top}$ is compactly generated;
	\item [(ii)]If for any compact Hausdorff space $S$ with a map $S\to X$, if the composition $S\to X\to Y$ is continuous, then $X\to Y$ is continuous;
	\item [(iii)] $A\subset X$ is closed if and only if for any compact space $K$ with a map $f:K\to X$, $f^{-1}(A)\subset K$ is closed.
\end{itemize}


\rem
\begin{itemize}
	\item [(i)]If a topological space $X$ is compact Hausdorff, then $X$ is compactly generated.
	\item [(ii)] Let $\text{CGTop}$ denote the category of all compactly generated spaces and let $\text{CHaus}$ denote the category of all compact Hausdorff spaces. 
\end{itemize}

\dfn 
For a category $\mathcal{C}$, $P\in \mathcal{C}$ is a projective object if for any epimorphism $Y\twoheadrightarrow X$ and a morphism $P\to X$, there is a lift
\begin{equation*}
\begin{tikzcd}
& P \arrow[d] \arrow[ld, "\exists"', dashed] \\
Y \arrow[r, two heads] & X                                         
\end{tikzcd}
\end{equation*}

\dfn 
In the category $\text{CHaus}$, we call its projective objects as extremally disconnected Hausdorff spaces.

\rem 
\begin{itemize}
	\item [(i)] Equivalently a compact Hausdorff space $S$ is extremally disconnected if any surjection $S^\prime\twoheadrightarrow S$ from a compact Hausdorff space splits.
	\item [(ii)] Extremally disconnected Hausdorff spaces are profinite sets, i.e. $\text{ExDisc}\subset \text{ProFin}$. Here, $\text{ExDisc}$ denote the category of all extremally disconnected Hausdorff spaces.
\end{itemize}

\rem We have two adjunctions.
\begin{itemize}
	\item [(i)] 
$\xymatrix{\text{Top} \ar@<0.5ex>[r]^-\beta
	& \text{CHaus}\ar@<0.5ex>[l]^-i }$, i.e. $\beta \dashv i$.\\
 Where $$i:\text{CHaus}\rightarrow \text{Top};\ X\mapsto X$$ and $$\beta:\text{Top}\rightarrow \text{CHaus}$$ is the Stone-Cech compactification of topological spaces.\\
For any $X\in\text{Top}$, we define $\beta X\in \text{CHaus}$ as follows:\\
for any $Y\in\text{CHaus}$ with a map $f:X\to Y$, there exists a unique map $\beta X\to Y$ so that the diagram commutes.
$$
\begin{tikzcd}
X \arrow[rd, "f"'] \arrow[rr, "i_X"] &   & \beta X \arrow[ld, "\exists !", dashed] \\
& Y &                                        
\end{tikzcd}
$$
In fact, we can use the ultrafilter to construct $\beta X$ concretely. And by this construction, we can show that 
$$|\beta X|\leq 2^{2^{|X|}}.$$
	\item [(ii)]
	$\xymatrix{\text{CGTop} \ar@<0.5ex>[r]^-i
		& \text{Top}\ar@<0.5ex>[l]^-c }$, i.e. $i \dashv c$.\\
	Where $$i:\text{CGTop}\rightarrow\text{Top};\ X\mapsto X$$ and
	$$c:\text{Top}\rightarrow\text{CGTop};\ X\mapsto X^{\text{cg}}.$$
	We define $X^{\text{cg}}$ as follows:
	\begin{itemize}
		\item As a set, $X^{\text{cg}}=X$.
		\item The topology of $X^{\text{cg}}$ is given by the quotient topology of 
		$$\coprod_{S\to X\ S\in \text{CHaus}} S\longrightarrow X.$$
	\end{itemize}
\end{itemize}
  
\prop 
\begin{itemize}
	\item [(i)] The functor $\text{Top}\rightarrow \text{Cond}_\kappa (\text{Set});\ T\mapsto \underline{T}$ is a faithful functor.
	\item [(ii)] When the above functor restricted to the full subcategory $\kappa\text{-CGTop}$ of all $\kappa$-compactly generated spaces, functor $\kappa\text{-CGTop}\rightarrow \text{Cond}_\kappa (\text{Set});\ T\mapsto \underline{T}$ is a fully faithful functor.
	\item [(iii)] The functor $\text{Top}\rightarrow \text{Cond}_\kappa (\text{Set});\ T\mapsto \underline{T}$ admits a left adjoint $\text{Cond}_\kappa (\text{Set})\rightarrow\text{Top};\ T\mapsto T(*)_\text{top}$. Here, $T(*)_\text{top}$ means the underlying set $T(*)$ equipped with the quotient topology of $\sqcup_{S\to T} S\rightarrow T(*)$, where the disjoint union runs over all $\kappa$-small profinite sets $S$ with a map to $T$, i.e. an element of $T(S).$ Moreover, we have $ \underline{T}(*)_\text{top}\cong T^{\kappa\text{-cg}}.$
\end{itemize}


\newpage
\section{Condensed abelian groups}
\dfn 
A condensed abelian group $T$ is a sheaf of abelian groups on $*_{\text{pro\'et}}$, i.e. a functor $T:*_{\text{pro\'et}}^{\text{op}}\to \text{Ab}$ satisfying the sheaf condition. And we denote the category of all condensed abelian groups by $\text{Cond(Ab)}.$
\dfn [Grothendieck's axioms] Let $\mathcal{C}$ be an abelian category.
~\
\begin{itemize}
	\item [(AB3)] All colimits exist.
	\item [(AB3*)]All limits exist.
	\item [(AB4)] Arbitrary direct sums are exact.
	\item [(AB4*)] Arbitrary products are exact.
	\item [(AB5)] Filtered colimits are exact.
	\item [(AB6)]  For any index set $J$ and filtered categories $I_j,\ j\in J$, with functors $I_j\rightarrow \text{Cond(Ab)};\  i\mapsto M_i$, the natural map
	$$
	\underset{(i_j\in I_j)_j}{\underrightarrow{\text{lim}}}\prod_{j\in J} M_{i_j}\longrightarrow \prod_{j\in J}\underset{i_j\in I_j}{\underrightarrow{\text{lim}}} M_{i_j}
	$$
	is an isomorphism.
\end{itemize}
\dfn
Let $\mathcal{C}$ be an abelian category. $M\in \mathcal{C}$ is compact if $\text{Hom}(M,-)$ commutes with filtered colimits, i.e. $\text{Hom}(M,\underset{i}{\underrightarrow{\text{lim}}}\ N_i)\cong \underset{i}{\underrightarrow{\text{lim}}}\ \text{Hom}(M,N_i).$

\thm
\begin{itemize}
	\item [(i)]$\text{Cond(Ab)}$ is an abelian category which satisfies Grothendieck's axioms (AB3), (AB4), (AB5), (AB6), (AB3*) and (AB4*).
	\item [(ii)]$\text{Cond(Ab)}$ is generated by compact projective objects.
\end{itemize}

\cor 
There is an adjunction:
$$
\xymatrix{\text{Cond}_\kappa (\text{Set}) \ar@<0.5ex>[r]
	& \text{Cond}_\kappa (\text{Ab})\ar@<0.5ex>[l]}.
$$
Where $\text{Cond}_\kappa (\text{Ab})\longrightarrow \text{Cond}_\kappa (\text{Set})$ is the forgetful functor and 
$$
\text{Cond}_\kappa (\text{Set})\longrightarrow \text{Cond}_\kappa (\text{Ab});\ T\mapsto \mathbb{Z}[T].
$$
Here, $\mathbb{Z}[T]:=(S\mapsto \mathbb{Z}[T(S)])^\text{sh}.$

\rem 
\begin{itemize}
	\item [(i)] For $S\in \text{ExDisc}$ and $M\in\text{Cond(Ab)}$, we have 
	$$\text{Hom}_{\text{Cond(Ab)}}(\mathbb{Z}[S],M)\cong \text{Hom}_{\text{Cond(Set)}}(\underline{S},M)\cong  M(S).$$
	Proof: We define the map:
	$$
	\mu:\text{Hom}_{\text{Cond(Set)}}(\underline{S},M)\longrightarrow M(S);\ \alpha\mapsto \alpha(S)(1_S),
	$$
	and the map
	$$
	\lambda:M(S)\longrightarrow \text{Hom}_{\text{Cond(Set)}}(\underline{S},M);\ x\mapsto \lambda(x),
	$$ where for $\lambda(x): \underline{S}\longrightarrow M$, 
	$$
	\lambda(x)(T): \text{Cont}(T,S)\longrightarrow M(T);\ f\mapsto M(f)(x).
	$$
	One can check that $\mu$ and $\lambda$ are inverse to each other, hence
	$$\text{Hom}_{\text{Cond(Set)}}(\underline{S},M)\cong  M(S).$$
	\qed
	
	
	
	\item [(ii)] For any $S\in \text{ExDisc}$, $\mathbb{Z}[S]\in \text{Cond(Ab)}$ is a compact and projective object.\\
	Proof:\\ 
	Compactness.
	$$
	\text{Hom}(\mathbb{Z}[S],\underrightarrow{\text{lim}}\ M_i)=(\underrightarrow{\text{lim}}\ M_i)(S)=\underrightarrow{\text{lim}}\ M_i(S)=\underrightarrow{\text{lim}}\ \text{Hom}(\mathbb{Z}[S],M_i).
	$$
	Projectiveness. For any exact sequence $M^\prime\rightarrow M\rightarrow M^{\prime\prime}$ in $\text{Cond(Ab)}$, the sequence
	$$M^\prime(S)\rightarrow M(S) \rightarrow M^{\prime\prime}(S)$$
	is exact, i.e.
	$$
	\text{Hom}(\mathbb{Z}[S], M^\prime)\rightarrow\text{Hom}(\mathbb{Z}[S], M)\rightarrow\text{Hom}(\mathbb{Z}[S], M^{\prime\prime})
	$$ is exact, so $\mathbb{Z}[S]$ is projective.
	\qed
	\item [(iii)]
	$\text{Cond}(\text{Ab})$ has enough projectives.
\end{itemize}






\prop 
We have two equivalences.
\begin{itemize}
	\item [(i)]$\text{Shv}(\kappa\text{-CHaus})\stackrel{\sim}{\longrightarrow}\text{Shv}(\kappa\text{-ProFin});\ T\mapsto T|_{\kappa\text{-ProFin}}.$
	\item [(ii)]$\text{Shv}(\kappa\text{-ProFin})\stackrel{\sim}{\longrightarrow}\text{Shv}(\kappa\text{-ExDisc});\ T\mapsto T|_{\kappa\text{-ExDisc}}.$
\end{itemize}

\rem 
In order a presheaf of sets $T$ to be a sheaf of sets, by definition, we need to check the sheaf condition in $\text{ProFin}$. Now, from the equivalence $\text{Shv}(\kappa\text{-ProFin})\stackrel{\sim}{\longrightarrow}\text{Shv}(\kappa\text{-ExDisc})$, we only need to check the sheaf condition in $\text{ExDisc}$. In this case, the condition(ii) is automatic:
 $T$ maps the pullback diagram
\begin{equation*}		
\begin{tikzcd}
S^\prime\times_{S} S^\prime \arrow[d, "p_1"'] \arrow[r, "p_2"] & S^\prime \arrow[d] \\
S^\prime \arrow[r]                                             & S                 
\end{tikzcd}
\end{equation*}
to a pullback diagram
\begin{equation*}	
\begin{tikzcd}
T(S^\prime\times_{S} S^\prime) & T(S^\prime) \arrow[l, "p_2^*"'] \\
T(S^\prime) \arrow[u, "p_1^*"] & T(S) \arrow[l] \arrow[u]       
\end{tikzcd}
\end{equation*}
This is because any cover of extremally disconnected sets splits. Specifically, the diagram
\begin{equation*}
\xymatrix{
	S^\prime\times_S S^\prime\ar[r] \ar[d] & S^\prime \ar[d] \\
	S^\prime\ar[r]^g &S\ar@/^9pt/[l]^f
}
\end{equation*}

can implies the following diagram:
\begin{equation*}
\xymatrix{
	T(S^\prime\times_S S^\prime)  & T(S^\prime) \ar[l] &\\
	T(S^\prime)\ar[u]\ar@/_9pt/[r]_{T(f)} &  T(S)\ar[l]_{T(g)} \ar[u]&\\
	 & &X \ar@/_9pt/[luu]\ar@/^9pt/[llu]\ar@{.>}[lu]_{\exists}
}
\end{equation*}
which means it is a pullback diagram.






\property 
There are some properties of the category $\text{Cond(Ab)}$ of condensed abelian groups.
\begin{itemize}
	\item [(i)]$\text{Cond(Ab)}$ has a symmetric monoidal tensor products $-\otimes -$, where for $M, N\in \text{Cond(Ab)}$, $M\otimes N=(S\mapsto M(S)\otimes N(S))^{\text{sh}}.$
	
	
	
	\item [(ii)] Functor $\text{Cond(Set)}\rightarrow \text{Cond(Ab)};\ T\mapsto \mathbb{Z}[T]$ is symmetric monoidal with respect to the product and the tensor product, i.e. $\mathbb{Z}[T_1\times T_2]=\mathbb{Z}[T_1]\otimes\mathbb{Z}[T_2].$\\
	Proof:
	
	
	
	
	\item [(iii)] For $T\in \text{Cond(Set)}$, $\mathbb{Z}[T]\in \text{Cond(Ab)}$ is flat.\\
	Proof: We need to show $-\otimes \mathbb{Z}[T]:\text{Cond(Ab)}\rightarrow \text{Cond(Ab)}$ is an exact functor.\\
	Take an exact sequence in $\text{Cond(Ab)}$:
	$$0\longrightarrow X\longrightarrow Y\longrightarrow Z\longrightarrow 0.$$
	For any $S\in\text{ExDisc}$, we have an exact sequence:
	$$
	0\longrightarrow X(S)\longrightarrow Y(S)\longrightarrow Z(S)\longrightarrow 0.
	$$
	Tensoring with the free abelian group $\mathbb{Z}[T(S)]$, we get an exact sequence:
	$$0\longrightarrow X(S)\otimes\mathbb{Z}[T(S)]\longrightarrow Y(S)\otimes\mathbb{Z}[T(S)]\longrightarrow Z(S)\otimes\mathbb{Z}[T(S)]\longrightarrow 0,$$
	i.e.
	$$
	0\longrightarrow(X\otimes \mathbb{Z}[T])(S)
	\longrightarrow(Y\otimes \mathbb{Z}[T])(S)
	\longrightarrow(Z\otimes \mathbb{Z}[T])(S)\longrightarrow 0.
	$$
	Hence the sequence 
	$$	0\longrightarrow X\otimes \mathbb{Z}[T]
	\longrightarrow Y\otimes \mathbb{Z}[T]
	\longrightarrow Z\otimes \mathbb{Z}[T] \longrightarrow 0.
	$$ exact and $\mathbb{Z}[T]$ is flat.
	\qed
	
	
	
	
	\item [(iv)] Given any $M, N\in \text{Cond(Ab)}$, we can give the group of homomorphisms $\text{Hom}(M, N)$ the structure of condensed abelian groups via the following definition, for any $S\in \text{ExDisc}$,
	$$\underline{\text{Hom}}(M,N)(S):=\text{Hom}(\mathbb{Z}[S]\otimes M, N).$$ 
	So we define an internal Hom-functor object.
	\item [(v)] There is an adjunction. For $P, M, N\in \text{Cond(Ab)}$, we have an isomorphism of abelian groups:
	$$\text{Hom}(P,\underline{\text{Hom}}(M,N))\cong \text{Hom}(P\otimes M, N). $$ 
	Proof:
	First, if $P=\mathbb{Z}[S]$ for some $S\in\text{ExDisc}$, then
	$$\text{Hom}(P,\underline{\text{Hom}}(M,N))=\text{Hom}(\mathbb{Z}[S],\underline{\text{Hom}}(M,N))=\underline{\text{Hom}}(M,N)(S)=\text{Hom}(\mathbb{Z}[S]\otimes M, N).$$
	Now, for general $P\in \text{Cond(Ab)}$, we can write $P=\underrightarrow{\text{lim}}\ \mathbb{Z}[S_i]$, so 
     \begin{align*}
     	\text{Hom}(P,\underline{\text{Hom}}(M,N))
     	&=\text{Hom}(\underrightarrow{\text{lim}}\ \mathbb{Z}[S_i],\underline{\text{Hom}}(M,N))\\
     	&=\underleftarrow{\text{lim}}\ \text{Hom}(\mathbb{Z}[S_i],\underline{\text{Hom}}(M,N))\\
     	&=\underleftarrow{\text{lim}}\ \text{Hom}(\mathbb{Z}[S_i]\otimes M, N)\\
     	&=\text{Hom}(\underrightarrow{\text{lim}}\ \mathbb{Z}[S_i]\otimes M, N)\\
     	&=\text{Hom}(P\otimes M, N).
     \end{align*}
	\qed

	
		
	\item [(vi)] As $\text{Cond(Ab)}$ has enough projectives, one can form the derived category $D(\text{Cond(Ab)})$. If $P\in \text{Cond(Ab)}$ is compact and projective, then $P[0]\in D(\text{Cond(Ab)})$ is a compact object of the dereived category, i.e. $\text{Hom}(P,-)$ commutes with arbitrary direct sums. In particular, $D(\text{Cond(Ab)})$ is compactly generated.
	\item [(vii)] Similarly, in the derived category $D(\text{Cond(Ab)})$, we have the adjunction:
	$$\text{Hom}(P,R\underline{\text{Hom}}(M,N))\cong \text{Hom}(P\otimes^L M, N).$$
	\item [(viii)] Let $\mathcal{D}(\text{Cond(Ab)})$ denote the derived $\infty$-category of $\text{Cond(Ab)}$ and $\mathcal{D}(\text{Ab})$ denote the derived $\infty$-category of $\text{Ab}$, then there is an equivalence
	$$\mathcal{D}(\text{Cond(Ab)})\cong \text{Cond}(\mathcal{D}(\text{Ab})).$$
\end{itemize}























\newpage
\section{$D(R)$}
\dfn
An $\infty$-category is a simplicial set $\mathcal{C}$ which satisfies the following extension condition:
\begin{equation*}
\begin{tikzcd}
\Lambda^n_i \arrow[r] \arrow[d, hook]      & \mathcal{C} \\
\Delta^n \arrow[ru, "\exists\ f"', dotted] &            
\end{tikzcd}
\quad\quad 0<\forall i<n.
\end{equation*}

\dfn 
Let $\mathcal{C}$ be an $\infty$-category. A zero object of $\mathcal{C}$ is an object which is both initial and final. We say that $\mathcal{C}$ is pointed if $\mathcal{C}$ contains a zero object.

\dfn 
Let $\mathcal{C}$ be a pointed $\infty$-category. A triangle in $\mathcal{C}$ is a diagram $\Delta^1 \times\Delta^1\rightarrow\mathcal{C}$ depicted as
\begin{equation*}
\begin{tikzcd}
X \arrow[r, "f"] \arrow[d] & Y \arrow[d, "g"] \\
0 \arrow[r]                & Z               
\end{tikzcd}
\end{equation*}
where 0 is a zero object in $\mathcal{C}$.\\
We say a triangle in $\mathcal{C}$ is a fiber sequence if it is a pullback and say a triangle in $\mathcal{C}$ is a cofiber sequence if it is a pushout.\\
We generally indicate a triangle by specifying only the pair of maps $X\stackrel{f}{\longrightarrow}Y\stackrel{g}{\longrightarrow}Z.$

\rem
Let $\mathcal{C}$ be a pointed $\infty$-category. A triangle in $\mathcal{C}$ consists of the following data:
\begin{itemize}
	\item [(i)] A pair of morphisms $f:X\to Y$ and $g:Y\to Z$ in $\mathcal{C}$.
	\item [(ii)] A 2-simplex in $\mathcal{C}$ corresponding to a diagram
	\begin{equation*}
	\begin{tikzcd}
	& Y \arrow[rd, "g"] &   \\
	X \arrow[ru, "f"] \arrow[rr, "h"] &                   & Z
	\end{tikzcd}
	\end{equation*}
	in $\mathcal{C}$, which identifies $h$ with the composition $g\circ f.$
	\item [(iii)]
	 A 2-simplex 
	 \begin{equation*}
	 \begin{tikzcd}
	 & 0 \arrow[rd] &   \\
	 X \arrow[ru] \arrow[rr, "h"] &              & Z
	 \end{tikzcd}
	 \end{equation*}
	 in $\mathcal{C}$, which we view as anullhomotopy of $h$.
\end{itemize}
\dfn 
Let $\mathcal{C}$ be a pointed $\infty$-category containing a morphism $g:X\longrightarrow Y$.\\
A fiber of $g$ is a fiber sequence
\begin{equation*}
\begin{tikzcd}
W \arrow[r ] \arrow[d] & X \arrow[d, "g"] \\
0 \arrow[r]                & Y               
\end{tikzcd}
\end{equation*}
and we denote $W=\text{fib}(g).$\\
Dually, a cofiber of $g$ is a cofiber sequence
\begin{equation*}
\begin{tikzcd}
X \arrow[r,"g" ] \arrow[d] & Y \arrow[d] \\
0 \arrow[r]                & Z               
\end{tikzcd}
\end{equation*}
and we denote $Z=\text{cofib}(g).$


\dfn 
An $\infty$-category $\mathcal{C}$ is stable if it satisfies the following conditions:
\begin{itemize}
	\item [(i)] There exists a zero object $0\in\mathcal{C}$.
	\item [(ii)] Every morphism in $\mathcal{C}$ admits a fiber and a cofiber.
	\item [(iii)] A triangle in $\mathcal{C}$ is a fiber sequence if and only if it is a cofiber sequence.
\end{itemize}
\rem
\begin{itemize}
	\item [(i)]For a stable $\infty$-category $\mathcal{C}$, we define the suspension functor $\Sigma:\mathcal{C}\longrightarrow\mathcal{C}$ and the loop functor $\Omega:\mathcal{C}\longrightarrow\mathcal{C}$ as follows:
	$$\Sigma(X):=\text{cofib}(X\longrightarrow 0)$$
	and
	$$\Omega(X):=\text{fib}(0\longrightarrow X).$$
	\item [(ii)]For a stable $\infty$-category $\mathcal{C}$, there is a homotopy equivalence:
	$$\text{Map}_\mathcal{C}(\Sigma X,Y)\stackrel{\sim}{\longrightarrow}\text{Map}_\mathcal{C}( X,\Omega Y).$$
	Besides, the unit $X\longrightarrow \Omega\Sigma(X)$and $\Sigma\Omega(Y) \longrightarrow Y$counit are isomorphic.
\end{itemize}
\dfn 
Let $R$ be a commutative ring, the $\infty$-category $D(R)$ is a stable $\infty$-category with all colimits, generated (as a cocomplete stable $\infty$-category) by a distinguished compact object 1, satisfying
$$
\pi_0\text{Map}(1,1)=R^{\text{op}},\quad\quad \pi_0\text{Map}(\Sigma^d 1,1)=0,\ \forall d\neq 0.
$$
For $X,Y\in D(R)$, we define
$$
[X,Y]:=\pi_0\text{Map}(X,Y)
$$
and 
$$[X,Y]_d:=[\Sigma^d X,Y]=[X,\Omega^d Y].$$
\rem
\begin{itemize}
	\item [(i)] From the definition of $D(R)$, we have
	$$[1,1]=\pi_0\text{Map}(1,1)=R^{\text{op}},\quad\quad  [1,1]_d=\pi_0\text{Map}(\Sigma^d 1,1)=0,\ \forall d\neq 0.$$
	\item [(ii)] If $d\geq 0$, we have
	$$[X,Y]_d=\pi_d\text{Map}(X,Y).$$
	\item [(iii)] Claim: In $D(R)$, for any integer $d$, we have $[X,Y]_d\in\text{Ab}$.\\
	Proof: First, if $d\geq 2$, $[X,Y]_d=\pi_d\text{Map}(X,Y)\in\text{Ab}.$
	For any $d\in \mathbb{Z}$, 
	$$[X,Y]_d=[\Sigma^d X,Y]=[\Sigma^{d-2} X,Y]_2\in\text{Ab}.$$
	\item [(iv)] For a stable $\infty$-category, a fiber sequence $X\rightarrow Y\rightarrow Z$ is at the same time a cofiber sequence, and vice versa. Hence, we will call it a fiber-cofiber sequence.
	\item [(v)] For a fiber-cofiber sequence $X\rightarrow Y\rightarrow Z$ in $D(R)$, we can induce a new fiber-cofiber sequence $Y\rightarrow Z\rightarrow \Sigma X$.
	\item [(vi)] Given a fiber-cofiber sequence $X\rightarrow Y\rightarrow Z$ and any $A\in D(R)$, we can induce two long exact sequences:
	$$
	\cdots\longrightarrow[A,X]_d\longrightarrow[A,Y]_d\longrightarrow[A,Z]_d\longrightarrow[A,X]_{d-1}\longrightarrow\cdots
	$$
	$$
	\cdots\longrightarrow[X,A]_{d+1}\longrightarrow[Z,A]_d\longrightarrow[Y,A]_d\longrightarrow[X,A]_{d}\longrightarrow\cdots.
	$$
	\item [(vii)] Assume
	\begin{equation*}
	\begin{tikzcd}
	A \arrow[d] \arrow[r] & B \arrow[d] \\
	C \arrow[r]           & D          
	\end{tikzcd}
	\end{equation*}
	is a pushout-pullback square in $D(R)$, then we can produce a triangle $A\rightarrow B\oplus C\rightarrow D$. With this, we can induce a long exact sequence.
\end{itemize}

\dfn
For any integer $d$, we define a functor $H_d:D(R)\rightarrow\text{Mod}_R;\ X\mapsto [1,X]_d.$
\rem
\begin{itemize}
	\item [(i)] We already know $[1,X]_d\in\text{Ab}$. And we need to show $[1,X]_d$ is an $R$-module.\\
	In fact, we have
	$$
	\text{Map}(\Sigma^d 1,\Sigma^d 1)\times\text{Map}(\Sigma^d 1,X)\rightarrow\text{Map}(\Sigma^d 1,X),
	$$
	applying the functor $\pi_0$, we get:
	$$
		\pi_0\text{Map}(\Sigma^d 1,\Sigma^d 1)\times\pi_0\text{Map}(\Sigma^d 1,X)=\pi_0(\text{Map}(\Sigma^d 1,\Sigma^d 1)\times\text{Map}(\Sigma^d 1,X))\rightarrow\pi_0(\text{Map}(\Sigma^d 1,X)),
	$$
	i.e.
	$$R^{\text{op}}\times
	[1,X]_d\rightarrow[1,X]_d,$$
	which implies $[1,X]_d\in \text{Mod}_R.$
    \item[(ii)] $H_d:D(R)\rightarrow\text{Mod}_R;\ X\mapsto [1,X]_d$ is a representable functor and $\Sigma^d 1$ represents $H_d$.
\end{itemize}

\lem 
\begin{itemize}
    \item [(i)]	$H_d(\prod_i X_i)=\prod_i H_d(X_i).$
	\item [(ii)] $H_d(\oplus_i X_i)=\oplus_i H_d(X_i).$	
	\item [(iii)] $H_d(\underset{\longrightarrow}{\lim}\ X_i)=\underset{\longrightarrow}{\lim}\ H_d(X_i).$	
	\item [(iv)] For a sequence of maps $\cdots\rightarrow X_n\rightarrow X_{n-1}\rightarrow\cdots$, we have a Milnor sequence:
	$$
	0\longrightarrow \underset{\longleftarrow}{\lim}^1 H_{d+1}(X_n)\longrightarrow H_d(\underset{\longleftarrow}{\lim}\ X_n)\longrightarrow \underset{\longleftarrow}{\lim}\ H_d(X_n)\longrightarrow 0.
	$$
\end{itemize}

\prop 
$f:X\rightarrow Y$ in $D(R)$ is an isomorphism if and only if $H_d(f):H_d(X)\stackrel{\sim}{\rightarrow} H_d(Y)$, for $\forall d\in\mathbb{Z}.$
\pf 
Take $Z=\text{cofib}(X\stackrel{f}{\rightarrow}Y)$, then $X\rightarrow Y\rightarrow Z$ is a fiber-cofiber sequence, and we can induce a long exact sequence
$$
\cdots\rightarrow H_d(X)\rightarrow H_d(Y)\rightarrow H_d(Z)\rightarrow H_{d-1}(X)\rightarrow\cdots.
$$
It suffices to show: if $Z\in D(R)$ with $H_d(Z)=0,\ \forall d\in\mathbb{Z}$, then $Z=0.$\\
Consider the full subcategory of $D(R)$:
$$
\mathcal{C}=\{A\in D(R)|\ [\Sigma^d A,Z]=0,\ \forall d\in\mathbb{Z}\}.
$$
Observe that:
\begin{itemize}
	\item $1\in\mathcal{C}$.
	\item $\mathcal{C}$ is stable under colimits. This is because
	$$[\Sigma^d \text{colim}\ A_i,Z]=[\text{colim}\ \Sigma^d A_i,Z]=\text{lim}\ [\Sigma^d A_i,Z]=0.$$
	\item $\mathcal{C}$ is stable under cofibers.
\end{itemize}
By definition of $D(R)$, we know $D(R)$ is generated as a cocomplete stable $\infty$-category by 1. Hence, $D(R)=\mathcal{C}$. Then by Yoneda's lemma, $Z=0.$
\qed

\prop \label{truncation}
Let $X\in D(R)$, then there exists $Y\in D(R)$ with a map $f:Y\rightarrow X$, s.t.
\begin{itemize}
	\item [(i)] $H_d(Y)=0,\ \forall d<0.$
	\item [(ii)] $H_d(f): H_d(Y)\stackrel{\sim}{\longrightarrow} H_d(X)$ are isomorphisms, $\forall d\geq 0.$
\end{itemize}
\pf 
We first prove: there exists a sequence of maps $Y_0\rightarrow Y_1\rightarrow Y_2\rightarrow\cdots$ in $D(R)_{/X}$, s.t. for any $n\geq 0$, $H_d(Y_n)=0,\ d<0$ and $H_d(Y_n\rightarrow X) $ are isomorphisms if $0\leq d<n$ and is a surjection if $d=n$.\\
We prove this by induction.\\
First, $n=0$. Let $Y_0=\oplus_I 1$ for $I=$ cartinal of $H_0(X)$,
 then the map $Y_0\to X$ can induce a surjection $H_0(Y_0)=R^{\oplus I}\twoheadrightarrow H_0(X)$ and for $d<0$, $H_d(Y_0)=0.$\\
 Now we assume that there exists a sequence
 $$Y_0\rightarrow Y_1\rightarrow Y_2\rightarrow\cdots\rightarrow Y_{n-1}$$
in $D(R)_{/X}$ satisfying the assumption.\\
Let $F=\text{fib}(Y_{n-1}\to X)$, then $F\to Y_{n-1}\to X$ is a fiber-cofiber sequence. We can find an index $I$, s.t. $\Sigma^{n-1}\oplus_I 1\rightarrow F$ can induce a surjection $H_{n-1}(\Sigma^{n-1}\oplus_I 1)\twoheadrightarrow H_{n-1}(F)$. Then let $Y_n=\text{cofib}(\Sigma^{n-1}\oplus_I 1\to F\to Y_{n-1})$, hence $\Sigma^{n-1}\oplus_I 1\to Y_{n-1}\to Y_n$ is also a fiber-cofiber sequence. Now, we check it satisfies the requirements.
\begin{itemize}
	\item [(a)] $d<0$.
	The fiber-cofiber sequence $\Sigma^{n-1}\oplus_I 1\to Y_{n-1}\to Y_n$ can induce a long exact sequence:
	$$
	\cdots\rightarrow H_{-1}(\Sigma^{n-1}\oplus_I 1)\rightarrow H_{-1}(Y_{n-1})\rightarrow H_{-1}(Y_{n})
	\rightarrow H_{-2}(\Sigma^{n-1}\oplus_I 1)\rightarrow H_{-2}(Y_{n-1})\rightarrow\cdots.
	$$
	Since for $k<0$, $H_k(Y_{n-1})=0$, we know $H_d(Y_n)=H_{d-1}(\Sigma^{n-1}\oplus_I 1)=0(d<0).$
	\item [(b)] First, there exists a map $Y_n\rightarrow X$, this is because
	\begin{equation*}
	\xymatrix
{
	\Sigma^{n-1}\oplus_I 1  \ar[r] \ar[d] & Y_{n-1}\ar@{=}[d] \ar[r] &Y_n\ar@{.>}[d]^{\exists}\\
	F\ar[r] & Y_{n-1}\ar[r] &X
}
	\end{equation*}
and $H_d(\Sigma^{n-1}\oplus_I 1)=0,$  $\forall  d\neq n-1$, then by
$$\cdots\rightarrow H_{n-2}(\Sigma^{n-1}\oplus_I 1)\rightarrow H_{n-2}(Y_{n-1})\rightarrow H_{n-2}(Y_{n})\rightarrow H_{n-3}(\Sigma^{n-1}\oplus_I 1)\rightarrow\cdots,$$
it implies that $0\leq \forall d\leq n-2,\ H_d(Y_n)\cong H_d(Y_{n-1})\cong H_d(X).$\\
We have the following diagram:
\begin{equation*}
\xymatrix@C=8pt@R=12pt
{
H_n(Y_{n-1})\ar[r] \ar@{=}[d]&H_n(Y_{n})\ar[r]\ar@{.>>}[d]&H_{n-1}(\Sigma^{n-1}\oplus_I 1)\ar[r]\ar@{->>}[d] &H_{n-1}(Y_{n-1})\ar[r]\ar@{=}[d] &H_{n-1}(Y_{n})\ar[r]\ar@{.>}[d]^{\sim} &H_{n-2}(\Sigma^{n-1}\oplus_I 1)\ar[r] \ar[d]^{\sim} &H_{n-2}(Y_{n-1})\ar@{=}[d]\\
H_n(Y_{n-1})\ar[r]&H_n(X)\ar[r]&H_{n-1}(F)\ar[r] &H_{n-1}(Y_{n-1})\ar[r] &H_{n-1}(X)\ar[r] &H_{n-2}(F)\ar[r]  &H_{n-2}(Y_{n-1})
}
\end{equation*}
By five's lemma, we can show $H_{n-1}(Y_n)\stackrel{\sim}{\rightarrow} H_{n-1}(X)$ and $H_n(Y_n)\twoheadrightarrow H_n(X).$\\
Now, for $Y_0\rightarrow Y_1\rightarrow Y_2\rightarrow\cdots$, we take $Y=\underrightarrow{\text{lim}}\ Y_n$, and hence we can get a map $Y\rightarrow X$.\\
By $H_d(\underrightarrow{\text{lim}}\ Y_n)=\underrightarrow{\text{lim}}\ H_d(Y_n)$, for $d<0$, $H_d(Y)=0$, and for $d\geq 0$,
$$
H_d(Y)=\underrightarrow{\text{lim}}\ (H_d(Y_0)\rightarrow H_d(Y_1)\cdots\rightarrow H_d(Y_{d+1})\rightarrow H_d(Y_{d+2})\rightarrow\cdots)=H_d(X).
$$
\qed
\end{itemize}

\prop 
For $X\in D(R)$, the following are equivalent:
\begin{itemize}
	\item [(i)] $H_d(X)=0$, $\forall d<0.$
	\item [(ii)] $X$ is generated by 1 under colimits.
	\item [(iii)] There exists a sequence of maps $X_0\rightarrow X_1\rightarrow X_2\rightarrow\cdots$ with $X=\underrightarrow{\text{lim}}\ X_i$, where for each $i$, the cofiber $\text{cofib}(X_{i-1}\rightarrow X_i)$ is of the form $\Sigma^i \oplus_I 1.$
\end{itemize}
\pf
(i)$\implies$ (iii). By previous proposition, for $X\in D(R)$, there exists a map $f:Y\rightarrow X$ with $H_d(Y)=0$ for $d<0$ and $H_d(f)$ are isomorphisms for $d\geq 0.$ Then, for $d<0$,
$$H_d(X)=H_d(Y)=0.$$
Hence, for any $d\in\mathbb{Z}$, $H_d(f)$ are isomorphisms. Thus, $f:Y\stackrel{\sim}{\rightarrow} X$.
\\From the construction of $Y$, we know there is a sequence of maps $X_0\rightarrow X_1\rightarrow X_2\rightarrow\cdots$ with $\underrightarrow{\text{lim}}\ X_i=Y\cong X.$ \\
By the fiber-cofiber sequence 
$\Sigma^{n-1}\oplus_I 1\to X_{n-1}\to X_n$, we can get a new fiber-cofiber sequence $  X_{n-1}\to X_n\to \Sigma^{n}\oplus_I 1$, i.e. $\text{cofib}(X_{n-1}\to X_n)=\Sigma^{n}\oplus_I 1$.\\
(iii)$\implies$ (ii).\\
We have $X_{i-1}\rightarrow X_i\rightarrow \Sigma^i \oplus_I 1$, which gives a new fiber-cofiber sequence: $\Sigma^{i-1} \oplus_I 1\rightarrow X_{i-1}\rightarrow X_i$. Then $X_i=\text{cofib}(\Sigma^{i-1} \oplus_I 1\rightarrow X_{i-1})=\text{colim}(0\leftarrow \Sigma^{i-1} \oplus_I 1\rightarrow X_{i-1}).$\\
Now, $X_1=\text{cofib}(\oplus_I 1\rightarrow X_0)=\text{cofib}(\oplus_I 1\rightarrow \oplus_J 1)=\text{colim}(0\leftarrow \oplus_I 1\rightarrow \oplus_J 1).$ Hence, each $X_i$ is generated by 1 under colimits. Finally, $X=\text{colim}\ X_i$ is also generated by 1 under colimits.\\
(ii)$\implies$ (i).\\
Arbitrary colimits can be written in terms of pushouts and filtered colimits. And $H_d$ commutes with filtered colimits. So it suffices to show that for $A, B, C$ with $H_d(A)=H_d(B)=H_d(C)=0,\ \forall d<0$, then for the pushout $D=\text{colim}(C\leftarrow A\rightarrow B)$, $H_d(D)=0,\ \forall d<0.$\\
This is because we can get a null-composite sequence $A\to B\oplus C\to D$, and induce a long exact sequence
$$
\cdots\rightarrow H_d(A)\rightarrow H_d(B)\oplus H_d(C)\rightarrow H_d(D)\rightarrow\cdots
$$ 
which implies $H_d(D)=0,\ \forall d<0.$
\qed
\newline
\newline
\dfn 
\begin{itemize}
	\item [(i)]$D(R)_{\geq 0}:=\{X\in D(R)\mid H_d(X)=0,\ \forall d<0\}.$
	\item [(ii)]$D(R)_{< 0}:=\{X\in D(R)\mid H_d(X)=0,\ \forall d\geq 0\}.$
	\item [(iii)] $\tau_{\geq 0}: D(R)\rightarrow D(R)_{\geq 0}; X\mapsto \tau_{\geq 0}(X):=Y$, which is constructed in Proposition \ref{truncation}.
\end{itemize}
~\\
Now, given any map $Z\rightarrow X$ in $D(R)$, we can get a commutative diagram:
$$
\begin{tikzcd}
\tau_{\geq 0}(Z) \arrow[d] \arrow[r, "\exists !", dashed] & \tau_{\geq 0}(X) \arrow[d] \\
Z \arrow[r]                                               & X                         
\end{tikzcd}
$$
Hence, $\tau_{\geq 0}: D(R)\rightarrow D(R)_{\geq 0}$ is a functor.
\prop 
$\xymatrix{D(R)_{\geq 0} \ar@<0.5ex>[r]^-i
	& D(R)\ar@<0.5ex>[l]^-{\tau_{\geq 0}} \ar@<0.5ex>[r]^-{\tau_{< 0}}
	& D(R)_{<0}\ar@<0.5ex>[l]^-i }$, i.e. $i \dashv \tau_{\geq 0}$ and $\tau_{<0}\dashv i$.

\cor 
For $X\in D(R)$, we have
$$
X\cong \underleftarrow{\text{lim}}\ \tau_{\leq n}(X)\quad\text{and}\quad 
X\cong \underrightarrow{\text{lim}}\ \tau_{\geq -n}(X).
$$
\pf 
We have a Milnor sequence:
$$
0\longrightarrow \underset{\longleftarrow}{\lim}^1 H_{d+1}(\tau_{\leq n}X)\longrightarrow H_d(\underset{\longleftarrow}{\lim}\ \tau_{\leq n}X)\longrightarrow \underset{\longleftarrow}{\lim}\ H_d(\tau_{\leq n}X)\longrightarrow 0.
$$
For $n\gg 0$, we have $H_d(\tau_{\leq n}X)=H_d(X)$, hence $ \underset{\longleftarrow}{\lim}\ H_d(\tau_{\leq n}X)=H_d(X).$ \\
And $\{H_{d+1}(\tau_{\leq n}X)\}_{n\in\mathbb{Z}}$ satisfies the Mittag-Leffler condition, hence $\underset{\longleftarrow}{\lim}^1 H_{d+1}(\tau_{\leq n}X)=0$. Therefore, from the above short exact sequence, we have
$$H_d(\underset{\longleftarrow}{\lim}\ \tau_{\leq n}X)\cong H_d(X),\ \forall d\in\mathbb{Z},$$ 
which impies $X\cong \underset{\longleftarrow}{\lim}\ \tau_{\leq n}X.$ \\
For another isomorphsim, from
$$H_d(\underrightarrow{\text{lim}}\ \tau_{\geq -n}X)
=\underleftarrow{\text{lim}}\ H_d(\tau_{\geq -n}X)=H_d(X),\ \forall d\in\mathbb{Z},
$$
one can show $X\cong \underrightarrow{\text{lim}}\ \tau_{\geq -n}(X).$
\qed

\dfn 
For any map $f:X\rightarrow Y$ in $D(R)$, we define its kernel to be $$\text{ker}(f):=\tau_{\geq 0} \text{fib}(X\to Y)$$ and its cokernel to be $$\text{coker}(f):=\tau_{\leq 0} \text{cofib}(X\to Y).$$

\prop 
Let $D(R)_0=\{X\in D(R)\mid H_d(X),\ \forall d\neq 0\}$.
\begin{itemize}
	\item [(i)]There is an isomorphism
	$
	H_0:D(R)_0\stackrel{\sim}{\longrightarrow}\text{Mod}_R.
	$
	\item [(ii)] Any object in $D(R)_0$ can be written as of the form $\text{coker}(\oplus_I 1\to \oplus_J 1).$
	\item [(iii)] $H_0:D(R)_0\longrightarrow\text{Mod}_R$ is an exact functor.
	\item [(iv)]$H_0:D(R)_0\longrightarrow\text{Mod}_R$ commutes with direct sums.
\end{itemize}
\pf 
\begin{itemize}
	\item [(ii)] For $X\in D(R)_0$, there exists $f: Y\to X$ with $H_d(Y)=0,\ \forall d<0$ and $H_d(f)$ are isomorphisms, $\forall d\geq 0$. \\
	By the construction of $Y_1$, $Y_1=\text{cofib}(\oplus_I 1\to \oplus_J 1).$ \\
	On the other hand, $X\cong \tau_{\leq 0} Y_1$. Hence,
	$$
	X\cong \tau_{\leq 0}\text{cofib}(\oplus_I 1\to \oplus_J 1)=\text{coker}(\oplus_I 1\to \oplus_J 1).
	$$
	\item [(iii)]
	In order to show that $H_0$ preserves exact sequences, it suffices to show $H_0$ preserves kernels and cokernels.\\
	For any map $f:X\to Y$ in $D(R)_0$, applying functor $\tau_{\geq 0}$ to sequence $\text{fib}(f)\to X\to Y$, we get a fiber-cofiber sequence 
	$$
	\text{ker}(f)=\tau_{\geq 0}\text{fib}(f)\to X\to Y.
	$$
	And it induces a long exact sequence
	$$
	0=H_1(Y)\rightarrow H_0(\text{ker}(f))\rightarrow H_0(X)\rightarrow H_0(Y)\rightarrow\cdots.
	$$
	Hence, $H_0(\text{ker}(f))=\text{ker}(H_0(X)\rightarrow H_0(Y)).$\\
	Dually ,we can prove $H_0(\text{coker}(f))=\text{coker}(H_0(X)\rightarrow H_0(Y)).$
\end{itemize}
\qed


\rem 
$1\in D(R)_0$ is compact and projective.\\
Proof:
Compactness is the definition.\\
For the projectiveness, we need to show that any epimorphism $X\twoheadrightarrow 1$ splits.\\
Let $F=\text{fib}(X\to 1)$. Consider $F\to X\to 1$.  Then $H_{-1}(F)=0.$\\
By $[M,N]_d=\text{Ext}_R^{-d}(M,N)$, we get $\text{Ext}_R^{-1}(1,F)=[1,F]_{-1}=H_{-1}(F)=0$. Hence $X\twoheadrightarrow 1$ splits.\qed
\newline
\newline
\newline
\dfn 
\begin{itemize}
	\item [(i)]A filtered object of $D(R)$ is an object in $\text{Fun}(\mathbb{Z}_{\leq }, D(R))$, i.e.
	$$\cdots\longrightarrow F(n-1)\rightarrow F(n)\rightarrow F(n+1)\rightarrow\cdots.$$
	\item [(ii)] A filtered object $F$ is convergent if $\underleftarrow{\text{lim}}\ F(n)=0.$
	\item [(iii)] $F(\infty):=\underrightarrow{\text{lim}}\ F(n).$ Call it the underlying object of $F$.
	\item [(iv)] The n-th associated graded $\text{gr}_n (F):=\text{cofib}(F(n-1)\to F(n))\stackrel{\triangle}{=}F(n)/F(n-1).$
\end{itemize}

~\\
Now, giving a convergent filtered object $F:\mathbb{Z}_{\leq }\rightarrow D(R)$, s.t. $\text{gr}_n (F)\in D(R)_n,\ \forall n$, we can define an $R$-module $M_n$:
$$
H_n: D(R)_n\rightarrow \text{Mod}_R; \text{gr}_n (F)\mapsto H_n(\text{gr}_n (F))\stackrel{\triangle}{=}M_n.
$$
From the sequence
$$
F(n-1)/F(n-2)\longrightarrow F(n)/F(n-2)\longrightarrow F(n)/F(n-1)\longrightarrow \Sigma(F(n-1)/F(n-2)),
$$
we get a map $d:H_n(\text{gr}_n (F))\longrightarrow H_n(\Sigma \text{gr}_{n-1} (F))$, i.e. $d:M_n\longrightarrow M_{n-1}$.\\
One can check $d^2=0$. Hence, given a convergent filtered object $F$, s.t. $\text{gr}_n (F)\in D(R)_n$, we define a chain complex of $R$-modules $M_*$.\\
We denote $\text{Fun}(\mathbb{Z}_{\leq }, D(R))_{\text{cx}}=\{F\in \text{Fun}(\mathbb{Z}_{\leq }, D(R))\mid F \text{ convergent} \}.$
\prop 
\begin{itemize}
	\item [(i)]$\text{Fun}(\mathbb{Z}_{\leq }, D(R))_{\text{cx}}\stackrel{\sim}{\longrightarrow} \text{Ch}_R;\ F\mapsto M_*.$
	\item [(ii)] $H_n(F(\infty))=H_n(M_*),\ \forall n.$
\end{itemize}















\newpage
\section{$D(\mathbb{Z})$}

\dfn 
Let $X\in\text{Top}.$ A sieve on $X$ is a set $\mathfrak{U}$ of open subsets of $X$, s.t. if $V\in\mathfrak{U}$ and $V^\prime \subset V$, then $V^\prime\in \mathfrak{U}$. If $U=\cup_{V\in\mathfrak{U}} V$, we say that the sieve $\mathfrak{U}$ covers U.

\dfn 
\begin{itemize}
	\item [(i)]Let $X\in \text{Top}$. Let $\mathcal{F}\in\text{PSh}(X,D(\mathbb{Z})$ be a presheaf with values in $D(\mathbb{Z})$, i.e. $\mathcal{F}\in\text{Fun}(\text{Op}(X)^\text{op}, D(\mathbb{Z}))$. We say $\mathcal{F}$ is a sheaf if for all sieves $\mathfrak{U}$ on $X$ covering $U\in\text{Op}(X)$, we have
	$$
	\mathcal{F}(U)\stackrel{\sim}{\longrightarrow} \underset{V\in\mathfrak{U}^\text{op}}{\underleftarrow{\text{lim}}}\ \mathcal{F}(V).
	$$
	\item [(ii)]For $U\in\text{Op}(X)$, one define $h_U\in\text{PSh}(X,D(\mathbb{Z}))$ via
		\begin{equation*}
		h_U(V)=
		\begin{cases}
		*& V\subset U \\
		\emptyset& \text{otherwise}		
		\end{cases}
		\end{equation*}
	\item [(iii)]For a sieve $\mathfrak{U}$, one define $h_\mathfrak{U}\in\text{PSh}(X,D(\mathbb{Z}))$ via
		\begin{equation*}
		h_\mathfrak{U}(V)=
		\begin{cases}
		*& V\in \mathfrak{U}\\
		\emptyset&  V\notin \mathfrak{U}	
		\end{cases}
		\end{equation*}
\end{itemize}


\prop 
Let $\mathcal{F}\in \text{PSh}(X,D(\mathbb{Z}))$, then $\mathcal{F}$ is a sheaf if and only if it satisfies:
\begin{itemize}
	\item [(i)] $\mathcal{F}(\emptyset)=*$.
	\item [(ii)] For any open subsets $V, V^\prime\in \text{Op}(X)$,
	$$
	\mathcal{F}(V\cup V^\prime)\stackrel{\sim}{\longrightarrow} \mathcal{F}(V)\times_{\mathcal{F}(V\cap V^\prime)} \mathcal{F}(V^\prime).
	$$
	\item [(iii)] For any sieve $\mathfrak{U}$, 
	$
	\mathcal{F}(\underset{V\in\mathfrak{U}}{\underrightarrow{\text{lim}}}\ V)\stackrel{\sim}{\longrightarrow} \underset{V\in\mathfrak{U}^\text{op}}{\underleftarrow{\text{lim}}}\ \mathcal{F}(V).
	$
\end{itemize}

\rem 
$$
\mathbb{Z}[h_U](V):=\mathbb{Z}[h_U(V)]=
\begin{cases}
\mathbb{Z}& V\subseteq U\\
0& V\nsubseteq U
\end{cases}
$$
$$
\mathbb{Z}[h_\mathfrak{U}](V):=\mathbb{Z}[h_\mathfrak{U}(V)]=
\begin{cases}
\mathbb{Z}& V\in \mathfrak{U}\\
0& V\notin \mathfrak{U}
\end{cases}
$$
$$
\text{Map}(\mathbb{Z}[h_U],\mathcal{F})=\text{Map}(\mathbb{Z},\mathcal{F}(U)).
$$
\begin{align*}
\text{Map}(\mathbb{Z}[h_\mathfrak{U}],\mathcal{F})
&=\underset{V\in\mathfrak{U}^\text{op}}{\underleftarrow{\text{lim}}}\ \text{Map}(\mathbb{Z}[h_V],\mathcal{F})\\
&=\underset{V\in\mathfrak{U}^\text{op}}{\underleftarrow{\text{lim}}}\ \text{Map}(\mathbb{Z},\mathcal{F}(V))\\
&=\text{Map}(\mathbb{Z},\underset{V\in\mathfrak{U}^\text{op}}{\underleftarrow{\text{lim}}}\ \mathcal{F}(V))\\
&=\text{Map}(\mathbb{Z},	\mathcal{F}(\underset{V\in\mathfrak{U}}{\underrightarrow{\text{lim}}}\ V)).
\end{align*}

\prop 
$\xymatrix
{\text{PSh}(X,D(\mathbb{Z}))\ar@<0.5ex>[r]^{\text{sh}}&
\text{Sh}(X,D(\mathbb{Z}))\ar@<0.5ex>[l]^{i}
}
$; $\mathcal{F}\mapsto \mathcal{F}^\text{sh}$. Moreover, $\mathcal{F}^\text{sh}=0$ iff $\mathcal{F}$ lies in the stable co-complete subcategory generated by $\text{cofib}(\mathbb{Z}[h_\mathfrak{U}]\to \mathbb{Z}[h_U])$ for all sieves $\mathfrak{U}$ covering $U$.

\dfn 
For $\mathcal{F}\in\text{PSh}(X,D(\mathbb{Z}))$, define $H_n(\mathcal{F})\in \text{PSh}(X,\text{Ab})$ by $H_n(\mathcal{F})(U)=H_n(\mathcal{F}(U)).$
\\
With this presheaf $H_n(\mathcal{F})\in\text{PSh}(X,\text{Ab})$, one can sheafify it to get a sheaf $H_n(\mathcal{F})^\text{sh}\in\text{Sh}(X,\text{Ab}).$

\prop 
Let $\mathcal{F}\in \text{PSh}(X,D(\mathbb{Z})).$
\begin{itemize}
	\item [(i)] If $\mathcal{F}^\text{sh}=0$, then $H_n(\mathcal{F})^\text{sh}=0,\ \forall n\in\mathbb{Z}.$
	\item [(ii)] If $\mathcal{F}$ is bounded above and $H_n(\mathcal{F})^\text{sh}=0,\ \forall n\in\mathbb{Z}$, then $\mathcal{F}^\text{sh}=0$.
\end{itemize}

\cor 
Let $\mathcal{F}\rightarrow \mathcal{G}$ be a map in $\text{PSh}(X,D(\mathbb{Z}))$.
\begin{itemize}
	\item [(i)] If $\mathcal{F}^\text{sh}\stackrel{\sim}{\longrightarrow} \mathcal{G}^\text{sh}$, then $H_n(\mathcal{F})^\text{sh}\stackrel{\sim}{\longrightarrow}H_n(\mathcal{G})^\text{sh},\ \forall n\in\mathbb{Z}.$
	\item [(ii)] If $\mathcal{F}$ and $\mathcal{G}$ are bounded above, and $H_n(\mathcal{F})^\text{sh}\stackrel{\sim}{\longrightarrow}H_n(\mathcal{G})^\text{sh},\ \forall n\in\mathbb{Z}$, then $\mathcal{F}^\text{sh}\stackrel{\sim}{\longrightarrow} \mathcal{G}^\text{sh}$.
\end{itemize}


\cor 
Let $\mathcal{F}\rightarrow \mathcal{G}$ be a map in $\text{PSh}(X,D(\mathbb{Z}))$ and $\mathcal{F}, \mathcal{G}$ are bounded above, then 
$$
\mathcal{F}^\text{sh}\stackrel{\sim}{\rightarrow} \mathcal{G}\quad \iff\quad 
\begin{cases}
\mathcal{G} \text{ is a sheaf.}\\
H_n(\mathcal{F})^\text{sh}\stackrel{\sim}{\rightarrow}H_n(\mathcal{G})^\text{sh},\ \forall n\in\mathbb{Z}.
\end{cases}
$$

\dfn 


\prop 









\newpage
\section{The t-structure on valued sheaves}

\dfn
A t-structure on a stable $\infty$-category $\mathcal{C}$ is a pair $(\mathcal{C}_{\geq0},\mathcal{C}_{\leq 0})$ of full sub-$\infty$-categories of $\mathcal{C}$ that are stable under equivalences and satisfy:
\begin{itemize}
	\item[(T1)] The suspension functor $\Sigma$ and the loop functor $\Omega$ restrict to $\mathcal{C}_{\geq0},\mathcal{C}_{\leq 0}$ resp. are fully faithful functors $\Sigma:\mathcal{C}_{\geq0}\to \mathcal{C}_{\geq0}$ and $\Omega:\mathcal{C}_{\leq0}\to \mathcal{C}_{\leq0}.$
	\item [(T2)] If $X\in \mathcal{C}_{\geq0}$ and $Y\in \mathcal{C}_{\leq0}$, then $\text{Map}(X,\Omega Y)\simeq *.$
	\item [(T3)] For every $X\in \mathcal{C}$, there exists a fiber sequence
	$$
	X^\prime\longrightarrow X\longrightarrow X^{\prime\prime}
	$$
	with $X^\prime \in \mathcal{C}_{\geq0}$ and $X^{\prime\prime} \in \mathcal{C}_{\leq -1}:=\Omega \mathcal{C}_{\leq0}.$  
\end{itemize}
We call $\mathcal{C}_{\geq0}$ and $\mathcal{C}_{\leq0}$ the connective and coconnective parts of the t-structure.
~\\

Given $n\in \mathbb{Z}$, we define $\mathcal{C}_{\geq n}:= \Sigma^n \mathcal{C}_{\geq0}\subset \mathcal{C}$ and $\mathcal{C}_{\leq n}:=\Sigma^n \mathcal{C}_{\leq0}\subset \mathcal{C}$, where for $n<0$, we have $\Sigma^n=\Omega^{-n}.$

The inclusions $i:\mathcal{C}_{\geq m}\rightarrow \mathcal{C}$ and $s:\mathcal{C}_{\leq n}\rightarrow \mathcal{C}$ admit adjoint functors
$$
\xymatrix{\mathcal{C}_{\geq m} \ar@<0.5ex>[r]^-i
	&\mathcal{C}\ar@<0.5ex>[l]^-{r}
	 \ar@<0.5ex>[r]^-{p}
	& \mathcal{C}_{\leq n}\ar@<0.5ex>[l]^-s }.
$$





In particular, the full sub-$\infty$-category $\mathcal{C}_{\geq m}\subset \mathcal{C}$ is closed under colimits, and the full sub-$\infty$-category $\mathcal{C}_{\leq n}\subset \mathcal{C}$ is closed under limits. From the adjoint pairs, we can form their counit and unit, and we get
$$
\tau_{\geq 0}X=(i\circ r)(X)\stackrel{\epsilon}{\longrightarrow} X\stackrel{\eta}{\longrightarrow} \tau_{\leq -1}X=(s\circ p)(X).
$$
The composition of the two maps is a point in the anima $\text{Map}(\tau_{\geq 0}X,\tau_{\leq -1}X)\simeq *$. So the composite map automatically admits a null-homotopy, which is unique, up to contractible ambiguity. We have the following commutative diagram:
$$
\xymatrix{
\mathcal{C}_{\leq m}\cap \mathcal{C}_{\geq n}   
                    \ar@<0.5ex>[r]^-i     \ar@<0.5ex>[d]^-s
& \mathcal{C}_{\leq m}  \ar@<0.5ex>[l]^-r    \ar@<0.5ex>[d]^-s \\
\mathcal{C}_{\geq n} \ar@<0.5ex>[r]^-i     \ar@<0.5ex>[u]^-p  
& \mathcal{C}       \ar@<0.5ex>[l]^-r      \ar@<0.5ex>[u]^-p 
}
$$
The canonical map
$$
p\circ r\stackrel{\eta\circ p\circ r}{\longrightarrow}r\circ i\circ p\circ r\simeq r\circ p\circ i\circ r\stackrel{r\circ p\circ \epsilon}{\longrightarrow} r\circ p
$$
is an equivalence.


We say the full sub-$\infty$-category
$$
\mathcal{C}^{\heartsuit}:=\mathcal{C}_{\geq 0}\cap \mathcal{C}_{\leq 0}\subset \mathcal{C}
$$ 
is the heart of the t-structure. For the functor
 $$\pi_0:=\tau_{\geq 0}\circ\tau_{\leq 0}\simeq\tau_{\leq 0}\circ\tau_{\geq 0}:\mathcal{C}\rightarrow\mathcal{C}^{\heartsuit},$$
 we call it the zeroth homotopy functor. The functor $\pi_0$ is additive, but is NOT exact. Instead, for all $n\in \mathbb{Z}$, we define 
$$
\pi_d:\mathcal{C}\longrightarrow\mathcal{C}^{\heartsuit}
$$
to be $\pi_d=\pi_0\circ \Omega^d$, and call it the $d$th homotopy functor. Now, a fiber sequence
$$
Z\stackrel{g}\longrightarrow Y\stackrel{f}\longrightarrow X
$$
in $\mathcal{C}$ gives rise to a long exact sequence
$$
\cdots\longrightarrow\pi_{d+1}(X)\longrightarrow\pi_d (Z)\longrightarrow\pi_d (Y)\longrightarrow\pi_d (X)\longrightarrow\cdots
$$
in the heart $\mathcal{C}^{\heartsuit}$.\\
If $f:Y\to X$ is an equivalence, then $f:\pi_d(Y)\to \pi_d(X)$ is an isomorphism for all $d\in\mathbb{Z}$, but the opposite is generally not the case.


~\\~\\
Now, for the stable $\infty$-category $D(\mathbb{Z})$, we defined homology functors $H_d:D(\mathbb{Z})\to \text{Mod}_{\mathbb{Z}}$ for all $d\in\mathbb{Z}$ by
$$
H_d(X)\simeq \pi_0\text{Map}(\Sigma^d 1,X)\simeq\pi_0\text{Map}(1,\Omega^d X).
$$

$D(\mathbb{Z})$ admits a t-structure $(D(\mathbb{Z})_{\geq 0},D(\mathbb{Z})_{\leq 0})$, where the connective part $D(\mathbb{Z})_{\geq 0}$ is spanned by those $X$ for which $H_d(X)\simeq 0$, for $d<0$, and the coconnective part $D(\mathbb{Z})_{\leq 0}$ is spanned by those $X$ for which $H_d(X)\simeq 0$, for $d>0.$ The zeroth homology functor
$$
H_0:D(\mathbb{Z})^{\heartsuit}\longrightarrow\text{Mod}_\mathbb{Z}
$$
is an equivalence of (abelian) categories. We have $H_d\simeq H_0\circ \pi_d$, so the functors $H_d$ and $\pi_d$ encode the same information.


\prop 
Let $X\in \text{Top}$, and let $\mathcal{C}$ be a stable $\infty$-category. A t-structure on $\mathcal{C}$ induces a t-structure on the stable $\infty$-category $\mathcal{P}(X,\mathcal{C})$ of $\mathcal{C}$-valued presheaves on $X$, where the coconnective part $\mathcal{P}(X,\mathcal{C})_{\leq 0}\simeq \mathcal{P}(X,\mathcal{C}_{\leq 0})$, and where the connective part $\mathcal{P}(X,\mathcal{C})_{\geq 0}$ is spanned by those $\mathcal{F}$ such that
$$
\text{Map}(\mathcal{F},\Omega\mathcal{G})\simeq *
$$
for all $\mathcal{G}\in \mathcal{P}(X,\mathcal{C}_{\leq 0}).$



~\\
~\\
A functor $f:\mathcal{D}\to\mathcal{C}$ between stable $\infty$-categories is exact iff it is left exact iff it is right exact.

An exact funcor $f:\mathcal{D}\to\mathcal{C}$ between stable $\infty$-categories with t-structures is left t-exact if $f(\mathcal{D}_{\leq 0})\subset \mathcal{D}_{\leq 0}$, and it is right t-exact if $f(\mathcal{D}_{\geq 0})\subset \mathcal{D}_{\geq 0}$. It is t-exact if it is both left t-exact and right t-exact. If $f:\mathcal{D}\to\mathcal{C}$ admits right adjoint functor $g:\mathcal{C}\to\mathcal{D}$, then $f$ is right t-exact iff $g$ is left t-exact.



\thm Let $X\in\text{Top}$ and $\mathcal{C}$ a presentable stable $\infty$-category.
\begin{itemize}
	\item [(1)] The sheafification functor $\text{ass}_X:\mathcal{P}(X,\mathcal{C})\to \text{Sh}(X,\mathcal{C})$ is t-exact, and the inclusion functor $\iota_X:\text{Sh}(X,\mathcal{C})\to  \mathcal{P}(X,\mathcal{C})$ is left t-exact.
	\item[(2)] The composite functor
	$$
	\text{Sh}(X,\mathcal{C}^\heartsuit)\stackrel{\iota_X^\heartsuit}{\longrightarrow}  \mathcal{P}(X,\mathcal{C}^\heartsuit)\simeq\mathcal{P}(X,\mathcal{C})^\heartsuit\stackrel{\text{ass}_X}{\longrightarrow} \text{Sh}(X,\mathcal{C})^\heartsuit
	$$
	is an equivalence of categories.
\end{itemize}

Write $\pi_0^p$ and $\pi_0^s$ for the homotopy functors associated with the t-structure on presheaves and sheaves. Since $\text{ass}_X$ is both exact and t-exact, we obtain a commutative square

\begin{center}
	\begin{tikzcd}
	{\mathcal{P}(X,\mathcal{C})} \arrow[r, "\pi_0^p"] \arrow[d, "\text{ass}_X"] & {\mathcal{P}(X,\mathcal{C})^\heartsuit} \arrow[d, "\text{ass}_X"] \\
	{\text{Sh}(X,\mathcal{C})} \arrow[r, "\pi_0^s"]                             & {\text{Sh}(X,\mathcal{C})^\heartsuit} 
	\centering                          
	\end{tikzcd}
\end{center}

























\newpage
\section{Sheaf}

\lem
If $\mathcal{A}$ is bounded above, i.e. $\exists d\in\mathbb{Z}$, s.t. $H_n (\mathcal{A})=0,$for all $n>d$, then $\mathcal{A}^{\text{sh}}$ is also bounded above.


\Q
For finite sets $X, X'$ with $X'\to X$ surjective and split, then 
$$
0\rightarrow\mathbb{Z}\left[X\right]\rightarrow\mathbb{Z}\left[X'\right]\rightarrow\mathbb{Z}\left[X'\times_{X} X'\right]\rightarrow\mathbb{Z}\left[X'\times_{X}X' \times_{X}X'\right]\rightarrow\cdots
$$
is exact.




















\lem
Arbitrary limits and filtered colimits preserves $D(\mathbb{Z})_{\leq d}$.
\pf
First we show $D(\mathbb{Z})_{\leq d}$ is closed under filtered colimits. Assume $X_i\in D(\mathbb{Z})_{\leq d}, i\in I$,then
$$
H_n(\underset{\longrightarrow}{\lim} X_i)=\underset{\longrightarrow}{\lim} H_n(X_i)=0, \ \text{for any } n>d.
$$
Hence $\underset{\longrightarrow}{\lim}X_i\in  D(\mathbb{Z})_{\leq d}.$
Then we show $ D(\mathbb{Z})_{\leq d}$ is closed under arbitrary limits. Assume $X_i\in  D(\mathbb{Z})_{\leq d}, n>d$, then

\begin{align*}
H_n(\lim X_i)
&=\left[\Sigma^{n} 1,\lim X_i\right] \\
&=\pi_0 \text{Map}(\Sigma^{n} 1,\lim X_i)\\
&=\pi_0 \lim \text{Map}(\Sigma^{n} 1,X_i)\\
&=\pi_0 \lim *\\
&=\pi_0  *\\
&=0.
\end{align*}
Hence, $\lim X_i\in D(\mathbb{Z})_{\leq d}. $
\qed

\prob
What is the relation between $\pi_n(\lim X_i)$ and $\lim \pi_n(X_i).$
Similarly, the relation between $\pi_n(\text{colim} X_i)$ and $\text{colim} \pi_n(X_i).$

\dfn
We define the singular homology functor to be the composite of 
$$
\text{Top}\rightarrow \text{Cond(Set)}\hookrightarrow \text{Cond(An)}\rightarrow \text{An},
$$
and denote it by $h: \text{Top}\rightarrow \text{An}$, where 
$\text{Top}\rightarrow \text{Cond(Set)}, X\mapsto \underline{X}; \text{Cond(An)}\rightarrow \text{An} $ is the left adjoint of $\text{An}\hookrightarrow \text{Cond(An)}$.

\dfn
For the forgetful functor $D(\mathbb{Z})_{\geq 0}\simeq \text{Ani(Ab)}\rightarrow \text{Ani(Set)}\simeq \text{An}$, it has a left adjoint, and we denote it by
$$
\mathbb{Z}\left[-\right]:\text{Ani(Set)}\rightarrow \text{Ani(Ab)};\ S\mapsto \mathbb{Z}\left[S\right].
$$

\dfn 
For $X\in \text{Top}$, we define its singular homology object to be
$$
\mathbb{Z}\left[h(X)\right]\in \text{Ani(Ab)}\simeq D(\mathbb{Z})_{\geq 0}\subset D(\mathbb{Z}).
$$



\lem Assume $\mathcal{A}\in \text{Sh}(X,D(\mathbb{Z}))$, $H_n(\mathcal{A})=0,\forall n>d$, and $H_d(\mathcal{A})\neq 0$, then $H_d(\mathcal{A})$ is a sheaf.
\pf
For $H_d(\mathcal{A})\in \text{PSh}(X,\text{Ab})$, we need to check $H_d(\mathcal{A})\in \text{Sh}(X,\text{Ab})$.

By denition, $H_d(\mathcal{A})(U)=H_d(\mathcal{A}(U))=H_d({\underset{\longleftarrow}{\lim}}\mathcal{A}(V)).$
By the Milnor's sequence, we have
$$
0\longrightarrow \underset{\longleftarrow}{\lim}^{1}H_{d+1}(\mathcal{A}(V))\longrightarrow H_d({\underset{\longleftarrow}{\lim}}\mathcal{A}(V))\longrightarrow \underset{\longleftarrow}{\lim}H_{d}(\mathcal{A}(V))\longrightarrow 0.
$$
Because $H_{d+1}(\mathcal{A})=0$, so the left term of this short exact sequence is 0, hence
$$
H_d(\mathcal{A})(U)=H_d({\underset{\longleftarrow}{\lim}}\mathcal{A}(V))=\underset{\longleftarrow}{\lim}H_{d}(\mathcal{A}(V))=\underset{\longleftarrow}{\lim}H_{d}(\mathcal{A})(V).
$$
Hence, $H_d(\mathcal{A})\in \text{Sh}(X,\text{Ab}).$
\qed


\prop 

Let $\mathcal{C}_0\subset \mathcal{C}$ be a full subcategory, then the following full subcategories of $\mathcal{C}$ agree:
\begin{itemize}
	\item the full subcategory generated under (small) colimits by $\mathcal{C}_0$;
	\item the full subcategory generated under filtered colimits and finite colimits by $\mathcal{C}_0$;
	\item the full subcategory generated under sifted colimits and finite produncts by $\mathcal{C}_0$.
\end{itemize}




\newpage
\section{Animation}
\thm[Yoneda] 
Let $\mathcal{C}$ be an $\infty$-category, the functor $$\mathcal{C}\hookrightarrow\text{Fun}(\mathcal{C}^{\text{op}},\text{An});\ X\mapsto (Y\mapsto \text{Hom}_{\mathcal{C}}(Y,X))$$ is fully faithful.

\rem
For $S$ to be an anima, we mean $S$ is an $\infty$-category; while $S$ to be a Kan complex, we mean $S$ is a 1-category.
~\\
~\\
Let $\mathcal{C}$ be a category which admits all small colimits.\\ Recall an object $X\in\mathcal{C}$ is compact (also called finitely presented) if $\text{Hom}(X,-)$ commutes with filtered colimits.\\
An object $X\in\mathcal{C}$ is projective if $\text{Hom}(X,-)$ commutes with reflexive coequalizers (coequalizers of parallel arrows $Y\rightrightarrows Z$ with a simultaneous section $Z\to Y$ of both maps).\\
Taken together, an object $X\in\mathcal{C}$ is compact projective if $\text{Hom}(X,-)$ commutes with filtered colimits and reflexive coequalizers, equivalently, $\text{Hom}(X,-)$ commutes with 1-sifted colimits.\\
~\\
Let $\mathcal{C}^\text{cp}\subset \mathcal{C}$ be the full subcategory of compact projective objects. There is a fully faithful embedding $\text{sInd}(\mathcal{C}^\text{cp})\longrightarrow \mathcal{C}.$\\
If $\mathcal{C}$ is generated under small colimits by  $\mathcal{C}^\text{cp}$, then the functor is an equivalence:
$$\text{sInd}(\mathcal{C}^\text{cp})\cong \mathcal{C}.$$
If $\mathcal{C}^\text{cp}$ is small, then
$$\text{sInd}(\mathcal{C}^\text{cp})\subset\text{Fun}((\mathcal{C}^\text{cp})^\text{op}, \text{Set})  $$
is exactly the full subcategory of functors that take finite coproducts in $\mathcal{C}^\text{cp}$ to products in $\text{Set}.$

\exm 
\begin{itemize}
	\item [(i)] If $\mathcal{C}=\text{Set}$, then $\mathcal{C}^\text{cp}=\text{FinSet}$, which generates $\mathcal{C}$ under small colimits.
	\item [(ii)]If $\mathcal{C}=\text{Ab}$, then $\mathcal{C}^\text{cp}=\text{FinFreeAb}$, which generates $\mathcal{C}$ under small colimits.
	\item [(iii)]If $\mathcal{C}=\text{Ring}$, then $\mathcal{C}^\text{cp}$=\{retracts of $\mathbb{Z}[X_1,\cdots,X_n]$\}, which generates $\mathcal{C}$ under small colimits.
	\item [(iv)]If $\mathcal{C}=\text{Cond(Set)}$, then $\mathcal{C}^\text{cp}=\text{ExDisc}$, which generates $\mathcal{C}$ under small colimits.
	\item [(v)]If $\mathcal{C}=\text{Cond(Ab)}$, then $\mathcal{C}^\text{cp}=\{\text{direct summands of }\mathbb{Z}[S]\mid S\in\text{ExDisc} \}$, which generates $\mathcal{C}$ under small colimits.
	\item [(vi)] $\mathcal{C}=\text{Cond(Ring)}$, then $\mathcal{C}^\text{cp}=\{\text{retracts of } \mathbb{Z}[\mathbb{N}[S]]\mid S\in\text{ExDisc}  \}$, which generates $\mathcal{C}$ under small colimits.
\end{itemize}

\dfn 
Let $\mathcal{C}$ be a category that admits all small colimits and
$\mathcal{C}$ is generated under small colimits by  $\mathcal{C}^\text{cp}$. The animation of $\mathcal{C}$ is the $\infty$-category $\text{Ani}(\mathcal{C})$ freely generated under sifted colimits by $\mathcal{C}^\text{cp}$.
 

\exm 
If $\mathcal{C}=\text{Set}$, then $\text{Ani}(\mathcal{C})=\text{Ani}(\text{Set})\stackrel{\triangle}{=}\text{Ani}$ is the $\infty$-category of animated sets, or anima in a short.\\
Any anima has a set of connected components, giving a functor $\pi_0:\text{Ani}\rightarrow\text{Set}$, which has a fully faithful right adjoint $\text{Set}\hookrightarrow\text{Ani}$.\\
Given an anima $A$ with a point $a\in A$ (meaning a map $a:*\to A$), one can define groups $\pi_i(A,a)$, for $i\geq 1$ and for $i\geq 2$, $\pi_i(A,a)\in\text{Ab}$.\\
An anima $A$ is $i$-truncated if $\pi_j(A,a)=0,\ \forall a\in A$ and $\forall j>i$. Then $A$ is 0-truncated if and only if it is in the essential image of $\text{Set}\hookrightarrow \text{Ani}.$\\
The inclusion of $i$-truncated anima into all anima has a left adjoint $\tau_{\leq i}.$ For all anima $A$, the natural map 
$$A\stackrel{\sim}{\longrightarrow} \text{lim}\ \tau_{\leq i}A$$ 
is an equivalence.\\
Picking any $a\in A$ and $i\geq 1$, the fiber of $\tau_{\leq i}A\rightarrow \tau_{\leq i-1}A$ over the image of $a$ is an Eilenberg-Maclane anima $K(\pi_i(A,a),i).$ Here, an Eilenberg-Maclane anima $K(\pi, i)$ with $i\geq 1$ and $\pi$ a group that is abelian if $i>0$, is a pointed connected anima with $\pi_j=0$ for $j\neq i$ and $\pi_i=\pi.$\\
In fact, the $\infty$-category of pointed connected anima $(A,a)$ with $\pi_j(A,a)=0$ for $j\neq i$ is equivalent to $\text{Grp}$ when $i=1$, and to $\text{Ab}$ when $i\geq 2.$

\rem 
There are several ways to describe $\text{Ani}(\mathcal{C}).$
\begin{itemize}
	\item [(i)] $\text{Ani}(\mathcal{C})$ is the full sub-$\infty$-category of objects in  $\text{Fun}((\mathcal{C}^\text{cp})^\text{op},\text{Ani})$
	taking finite disjoint unions to finite products.
	\item [(ii)] $\text{Ani}(\mathcal{C})$ is the $\infty$-category obtained from $\text{Simp}(\mathcal{C})$ by inverting weak equivalences.
\end{itemize}


\dfn 
Let $\mathcal{C}$ be an $\infty$-category that admits all small colimits. For any uncountable strong limit cardinal $\kappa$, the $\infty$-category $\text{Cond}_\kappa (\mathcal{C})$ of $\kappa$-condensed objects of $\mathcal{C}$ is the category of contravariant functors from $\kappa\text{-ExDisc}$ to $\mathcal{C}$ that take finite coproducts to finite products.\\
And we define
$$
\text{Cond}(\mathcal{C}):=\bigcup_\kappa \text{Cond}_\kappa (\mathcal{C}).
$$
\prop 
Let $\mathcal{C}$ be a category that is generated under small colimits by $\mathcal{C}^\text{cp}$. Then $\text{Cond}(\mathcal{C})$ is still generated under small colimits by its compact projective objects, and there is a natural equivalence of $\infty$-categories
$$\text{Cond}(\text{Ani}(\mathcal{C}))\cong \text{Ani}(\text{Cond}(\mathcal{C})).$$
\newline
\dfn 
Let $\mathcal{C}$ be some site. 
\begin{itemize}
	\item [(i)]A presheaf of anima is a functor
	$$
	\mathcal{F}:\mathrm{N}(\mathcal{C}^\text{op})\longrightarrow \text{Ani}.
	$$
	\item [(ii)] A sheaf of anima is a presheaf of anima $	\mathcal{F}$, s.t. for all coverings $\{f_i:X_i\to X\}_{i\in I}$, one has
	$$
	\mathcal{F}(X)\stackrel{\sim}{\longrightarrow} 
	\text{lim} ( 
	\xymatrix{
	\prod_i \mathcal{F}(X_i) \ar@<0.5ex>[r]\ar@<-0.5ex>[r] &\prod_{i,j} \mathcal{F}(X_i\times_X X_j)\ar@<0.8ex>[r]\ar@<-0.8ex>[r]\ar[r]&\cdots
     }).
 	$$
 	\item [(iii)] A hypercomplete sheaf of anima is a sheaf of anima $\mathcal{F}$, s.t. for all hypercovers $X_\bullet\rightarrow X$, the map
 	$$
 	\mathcal{F}(X)\stackrel{\sim}{\longrightarrow}\text{lim}\ \mathcal{F}(X_\bullet)=\text{lim}\ (
 	\xymatrix{
 		 \mathcal{F}(X_0) \ar@<0.5ex>[r]\ar@<-0.5ex>[r] & \mathcal{F}(X_1)\ar@<0.8ex>[r]\ar@<-0.8ex>[r]\ar[r]&\cdots
 	}
 	)
 	$$
 	is an equivalence.
\end{itemize}


\dfn 
The $\infty$-category of condensed anima is given by 
\begin{itemize}
	\item [-] The $\infty$-category of hypercomplete sheaves of anima on $\text{CHaus}.$
	\item [-] The $\infty$-category of hypercomplete sheaves of anima on $\text{ProFin}.$
	\item [-] The $\infty$-category of hypercomplete sheaves of anima on $\text{ExDisc}$, i.e. of functors
	$$\text{ExDisc}^\text{op}\longrightarrow \text{Ani}$$
	taking finite disjoint unions to finite products.
\end{itemize}
~\\
~\\
\begin{equation*}
\xymatrix@=18pt{
	\text{CW}\ar@{}[d]|{\bigcap}  & \subset &\text{Cond(Set)}\ar@{}[d]|{\bigcap}\\
	\text{Ani}& \subset &\text{Cond(Ani)}
}
\end{equation*}
\dfn 
$X\in \text{Cond(Ani)}$ is
\begin{itemize}
	\item [-] discrete, if $X$ in the essential image of $\text{Ani}.$
	\item [-] static, if $X$ in the essential image of $\text{Cond(Set)}.$
\end{itemize}












\newpage
\section{Condensed Cohomology}

\dfn
Let $X\in \text{Cond}, M\in\text{Cond(Ab)}$, we define the global section of $M$ on $X$ to be  
$$
\Gamma_{\text{cond}}(X,M):=\text{Hom}_{\Cond}(X,M)=\text{Hom}_{\text{Cond(Ab)}}(\mathbb{Z}\left[X\right],M)\in\text{Ab},
$$
and we define the condensed cohomology to be 
$$
R\Gamma_{\text{cond}}(X,M):=R\text{Hom}_{\text{Cond(Ab)}}(\mathbb{Z}\left[X\right],M),
$$ i.e.
$$
H^i_{\text{cond}}(X,M):=\text{Ext}^i_{\text{Cond(Ab)}}((\mathbb{Z}\left[X\right],M).
$$

\lem\label{exdisc}
For $X\in \text{ExDisc}$, the functor $\Gamma_{\text{cond}}(X,-):\text{Cond(Ab)}\to \text{Ab}$ is exact, hence, for any $M\in \text{CondAb}, H^i_{\text{cond}}(X,M)=0, \forall i\geq 1.$
\pf
We have $\Gamma_{\text{cond}}(X,-)=\text{Hom}_{\text{Cond(Ab)}}(\mathbb{Z}\left[X\right],-)$, and for $X\in \text{ExDisc}$, $\mathbb{Z}\left[X\right]$ is projective, hence $\Gamma_{\text{cond}}(X,-):\text{Cond(Ab)}\to \text{Ab}$ is exact.
\qed

~\\
\Q
How to compute $H^i_{\text{cond}}(X,M)$?

From the definition, we need to find a projective resolution of $\mathbb{Z}\left[X\right]$.

For $X\in\text{CHaus}$, we pick a hypercover $X_{\bullet}\to X$, where each $X_i\in \text{ExDisc}$, for this hypercover, applying $\mathbb{Z}\left[-\right]$, then we get a projective resolution of $\mathbb{Z}\left[X\right]$:
$$
\cdots\longrightarrow\mathbb{Z}\left[X_2\right]\longrightarrow\mathbb{Z}\left[X_1\right]\longrightarrow\mathbb{Z}\left[X_0\right]\longrightarrow\mathbb{Z}\left[X\right]\longrightarrow 0.
$$
By definition, we have
\begin{align*}
H^i_{\text{cond}}(X,M)&=\text{Ext}^i_{\text{Cond(Ab)}}(\mathbb{Z}\left[X\right],M)\\
&=H^i(0\rightarrow\text{Hom}_{\text{Cond(Ab)}}(\mathbb{Z}\left[X_0\right],M)\rightarrow\text{Hom}_{\text{Cond(Ab)}}(\mathbb{Z}\left[X_1\right],M)\rightarrow\cdots)\\
&=H^i(0\rightarrow\Gamma_{\text{cond}}(X_0,M)\rightarrow\Gamma_{\text{cond}}(X_1,M)\rightarrow\Gamma_{\text{cond}}(X_2,M)\rightarrow\cdots).
\end{align*}

\thm[Dyckhoff,1976]
For any $X\in \text{CHaus}$, there are natural isomorphisms: 
$$
H^i_{\text{cond}}(X,\mathbb{Z})\cong H^i_\text{sh}(X,\mathbb{Z}),\ \forall i\geq 0. 
$$
\pf
\begin{itemize}
	\item[1)] 
	Assume $X\in \text{Fin}$, then
	\begin{equation*}
	H^i_{\text{cond}}(X,\mathbb{Z})=\left\{
	\begin{aligned}
	&\Gamma_{\text{cond}}(X,\mathbb{Z})=C(X,\mathbb{Z}) \quad &i=0\\
	&0 \quad &i>0
	\end{aligned}
	\right.
	\end{equation*}
This comes from Lemma \ref{exdisc}. On the other hand,
  	\begin{equation*}
  	H^i_{\text{sh}}(X,\mathbb{Z})=\check{H}^i(X,\mathbb{Z})=\left\{
  	\begin{aligned}
  	&C(X,\mathbb{Z}) \quad &i=0\\
  	&0 \quad &i>0
  	\end{aligned}
  	\right.
  	\end{equation*}
  	This comes from by computing Cech cohomology. For a finite set $X$, take the cover $\mathcal{U}=\{x\to X\}_{x\in X}$, then 
  	$\mathcal{C}^0(\mathcal{U},\mathbb{Z})=\mathcal{C}^1(\mathcal{U},\mathbb{Z})=\cdots =\mathbb{Z}^X$, and because $\mathcal{U}$ is a refinement of any cover, we have
  	\begin{equation*}
    \check{H}^i(X,\mathbb{Z})=\check{H}^i(\mathcal{U},\mathbb{Z})=\left\{
  	\begin{aligned}
  	&\mathbb{Z}^X=C(X,\mathbb{Z}) \quad &i=0\\
  	&0 \quad &i>0
  	\end{aligned}
  	\right.
  	\end{equation*}
  Therefore, for a finite set $X$, $H^i_{\text{cond}}(X,\mathbb{Z})\cong H^i_\text{sh}(X,\mathbb{Z}),\ \forall i\geq 0.$
  	\item[2)] 
  	$X\in \text{ProFin}$, hence we can write $X=\underleftarrow{\text{lim}}_j X^j, \ X^j\in \text{Fin}.$
  	\begin{equation*}
  		H^i_\text{sh}(X,\mathbb{Z})=\check{H}(X,\mathbb{Z})=\underrightarrow{\text{lim}}_j \check{H}(X_j,\mathbb{Z})=\left\{
  	\begin{aligned}
  	&\underrightarrow{\text{lim}}_j C(X_j,\mathbb{Z})=C(X,\mathbb{Z}) \quad &i=0\\
  	&0 \quad &i>0
  	\end{aligned}
  	\right.
  	\end{equation*}
  	On the other hand, We compute $H^i_{\text{cond}}(X,\mathbb{Z}),\ i\geq 0.$ \\
  	For $X\in\text{ProFin}$, pick a hypercover $X_\bullet\to X$ with each $X_i\in\text{ExDisc}$, and for each $X^j$, pick a finite hypercover $X^j_\bullet\to X^j$, s.t. $\underleftarrow{\text{lim}}_j X_n^j=X_n$. Since $X^j$ is finite, we have 
  	\begin{equation*}
  	H^i_{\text{cond}}(X^j,\mathbb{Z})=\left\{
  	\begin{aligned}
  	&\Gamma(X^j,\mathbb{Z}) \quad &i=0\\
  	&0 \quad &i>0
  	\end{aligned}
  	\right.
  	\end{equation*}
  	And we know
  	$$H^i_{\text{cond}}(X^j,\mathbb{Z})=H^i(0\longrightarrow \Gamma(X_0^j,\mathbb{Z})\longrightarrow \Gamma(X_1^j,\mathbb{Z})\longrightarrow \Gamma(X_2^j,\mathbb{Z})\rightarrow\cdots),$$
  	hence we have an exact sequence:
  	$$0\longrightarrow \Gamma(X^j,\mathbb{Z})\longrightarrow \Gamma(X_0^j,\mathbb{Z})\longrightarrow \Gamma(X_1^j,\mathbb{Z})\longrightarrow \Gamma(X_2^j,\mathbb{Z})\rightarrow\cdots.$$
  	Applying the exact functor $\underset{j}{\underrightarrow{\text{lim}}}$ to this exact sequence, we get an exact sequence:
  	$$
  	0\longrightarrow \underset{j}{\underrightarrow{\text{lim}}}\ \Gamma(X^j,\mathbb{Z})\longrightarrow \underset{j}{\underrightarrow{\text{lim}}}\ \Gamma(X_0^j,\mathbb{Z})\longrightarrow \underset{j}{\underrightarrow{\text{lim}}}\ \Gamma(X_1^j,\mathbb{Z})\longrightarrow \underset{j}{\underrightarrow{\text{lim}}}\ \Gamma(X_2^j,\mathbb{Z})\rightarrow\cdots,
  	$$
  	i.e.
  	$$
  	0\longrightarrow \Gamma(X,\mathbb{Z})\longrightarrow \Gamma(X_0,\mathbb{Z})\longrightarrow \Gamma(X_1,\mathbb{Z})\longrightarrow \Gamma(X_2,\mathbb{Z})\rightarrow\cdots.
  	$$
  	Hence,
  	\begin{equation*}
  	H^i_{\text{cond}}(X,\mathbb{Z})=\left\{
  	\begin{aligned}
  	&\Gamma(X,\mathbb{Z}) \quad &i=0\\
  	&0 \quad &i>0
  	\end{aligned}
  	\right.
  	\end{equation*}
  	\item [3)] $X\in\text{CHaus}.$\\
  	Consider a morphism of topoi $(\alpha^{-1},\alpha_*):\text{Sh}(\text{CHaus}/X)\to \text{Sh}(X)$. For $\mathcal{F}\in \text{Sh}(\text{CHaus}/X)$, $\alpha_*\mathcal{F}$ is given by 
  	$$U\mapsto \underset{V\subset U,\ V \text{ is closed in } X}{\underleftarrow{\text{lim}} }\mathcal{F}(V\hookrightarrow S).$$ 
  	We have the following diagram:
  	\begin{equation*}
  	\begin{tikzcd}
  	\text{Sh}(\text{CHaus}/X) \arrow[rr, "\alpha_*"] \arrow[rd, "{\Gamma_{\text{cond}}(X,-)}"'] &            & \text{Sh}(X) \arrow[ld, "{\Gamma_{\text{sh}}(X,-)}"] \\
  	& \text{Set} &                                                     
  	\end{tikzcd}
  	\end{equation*}
  	This is because $\forall\  Y\in \text{Sh}(\text{CHaus}/X)$,
  	\begin{align*}
  	\Gamma_{\text{sh}}(X,\alpha_* Y)
  	&=\alpha_* Y(X)=\underset{V\subset U,\ V \text{ is closed in } X}{\underleftarrow{\text{lim}} } Y(V)\\
  	&=\underset{V}{\underleftarrow{\text{lim}}}\  \text{Hom}_\text{cond}(V,Y)=\text{Hom}_\text{cond}(\underset{V}{\underrightarrow{\text{lim}}} \ V,Y)\\
  	&=\text{Hom}_\text{cond}(X,Y)=\Gamma_{\text{cond}}(X,Y).
  	\end{align*}
  	And this diagram can induce a diagram:
  	\begin{equation*}
  	\begin{tikzcd}
  	D(\text{Ab}(\text{CHaus}/X)) \arrow[rr, "R\alpha_*"] \arrow[rd, "{R\Gamma_{\text{cond}}(X,-)}"'] &            & D(\text{Ab}(X)) \arrow[ld, "{R\Gamma_{\text{sh}}(X,-)}"] \\
  	& D(\text{Ab}) &                                                     
  	\end{tikzcd}
  	\end{equation*}
  	Claim: $R\alpha_*\mathbb{Z}\cong \mathbb{Z}$ in $D(\text{Ab}(X))$.\\
  	With this claim, we can show 
  	\begin{align*}
  	H^i_{\text{cond}}(X,\mathbb{Z})
  	&=H^i(R\Gamma_{\text{cond}}(X,\mathbb{Z}))\\
  	&=H^i(R\Gamma_{\text{sh}}(X,-)\circ R\alpha_* \mathbb{Z})\\
  	&=H^i(R\Gamma_{\text{sh}}(X,\mathbb{Z}))\\
  	&=H^i_\text{sh}(X,\mathbb{Z}).
  	\end{align*}
  	Hence, it suffices to show this claim. We have a map $\mathbb{Z}\to  R\alpha_*\mathbb{Z}$ in $D(\text{Ab}(X))$. In order to show this is an isomorphism, it suffices to check on each stacks.\\
  	Fix $s\in S$,
  	\begin{align*}
  	(R\alpha_*\mathbb{Z})_s
  	&=\underset{s\in U\ \text{open}}{\underrightarrow{\text{lim}}}R\Gamma(U,R\alpha_*\mathbb{Z})\\
  	&=\underset{s\in U\ \text{open}}{\underrightarrow{\text{lim}}} R\Gamma_{\text{cond}}(U,\mathbb{Z})\\
  	&=\underset{s\in V\ \text{closed}}{\underrightarrow{\text{lim}}}R\Gamma_{\text{cond}}(V,\mathbb{Z}).
  	\end{align*}
  	Pick a hypercover $S_\bullet\to S$ with $S_i\in\text{ExDisc}$. Then for each closed $V$, $(S_n\times_{X} V)_{n\geq 0}\to V$ is a hypercover. Hence,
  	$$R\Gamma_{\text{cond}}(V,\mathbb{Z})\cong (0\longrightarrow \Gamma(S_0\times_{X} V,\mathbb{Z})\longrightarrow \Gamma(S_1\times_{X} V,\mathbb{Z})\longrightarrow\cdots).$$
  	Thus, we have
  	\begin{align*}
  	(R\alpha_*\mathbb{Z})_s
  	&=\underset{s\in V\ \text{closed}}{\underrightarrow{\text{lim}}}R\Gamma_{\text{cond}}(V,\mathbb{Z})\\
  	&\cong\underset{s\in V\ \text{closed}}{\underrightarrow{\text{lim}}}(0\longrightarrow \Gamma(S_0\times_{X} V,\mathbb{Z})\longrightarrow \Gamma(S_1\times_{X} V,\mathbb{Z})\longrightarrow\cdots)\\
  	&\cong(0\longrightarrow \underset{s\in V\ \text{closed}}{\underrightarrow{\text{lim}}}\Gamma(S_0\times_{X} V,\mathbb{Z})\longrightarrow\underset{s\in V\ \text{closed}}{\underrightarrow{\text{lim}}}\Gamma(S_1\times_{X} V,\mathbb{Z})\longrightarrow\cdots)\\
  	&\cong(0\longrightarrow \Gamma(S_0\times_{X} \{s\},\mathbb{Z})\longrightarrow \Gamma(S_1\times_{X} \{s\},\mathbb{Z})\longrightarrow\cdots)\\
  	&\cong R\Gamma_{\text{cond}}(\{s\},\mathbb{Z})\\
  	&\cong\mathbb{Z},
  	\end{align*}
  	which finishes our proof.
  	\qed
\end{itemize}
	
\exm
Let $\mathbb{T}=\mathbb{R}/\mathbb{Z}$, for $\mathbb{T}^I\in\text{CHaus}$, we have $H^n(\mathbb{T}^I,\mathbb{Z})=\wedge^n(\mathbb{Z}^{\oplus I}).$
\pf
First, we have
\begin{equation*}
H^n(\mathbb{T},\mathbb{Z})=\left\{
\begin{aligned}
&\mathbb{Z} \quad &n=0, 1\\
&0 \quad &\text{else}
\end{aligned}
\right.
\end{equation*}
i.e. $H^*(\mathbb{T},\mathbb{Z})=\wedge(\mathbb{Z}).$~\\
Claim: $H^*(\mathbb{T}^n,\mathbb{Z})=\wedge(\mathbb{Z}^{\oplus n}).$~\\
We can prove it by induction on $n$. $n=1$ is proved above.\\
By Kunneth theorem, we can show that for $H^*(X,\mathbb{Z})$ finitely generated free in each degree, we have $H^*(X\times Y,\mathbb{Z})\cong H^*(X,\mathbb{Z})\otimes H^*(Y,\mathbb{Z})$. Hence, we have
\begin{align*}
H^*(\mathbb{T}^n,\mathbb{Z})
&= H^*(\mathbb{T}^{n-1},\mathbb{Z})\otimes H^*(\mathbb{T},\mathbb{Z})\\
&=\wedge(\mathbb{Z}^{\oplus (n-1)})\otimes \wedge(\mathbb{Z})\\
&=\wedge(\mathbb{Z}^{\oplus n}).
\end{align*}
In order to prove the general case, there is a fact that for $S\in \text{CHaus}$, $S=\underset{j}{\underleftarrow{\text{lim}}} S_j$, then $
H^n(S,\mathbb{Z})=\underset{j}{\underrightarrow{\text{lim}}}\  H^n(S_j,\mathbb{Z}).$\\
Hence,
\begin{align*}
H^n(\mathbb{T}^I,\mathbb{Z})
&=H^n(\underset{J\subset I\ \text{ finite}}{\underleftarrow{\text{lim}}}\ \mathbb{T}^J,\mathbb{Z})\\
&=\underset{J\subset I\ \text{ finite}}{\underrightarrow{\text{lim}}}\  H^n(\mathbb{T}^J,\mathbb{Z})\\
&=\underset{J\subset I\ \text{ finite}}{\underrightarrow{\text{lim}}}\ \wedge^n(\mathbb{Z}^{\oplus J})\\
&=\wedge^n(\mathbb{Z}^{\oplus I}).
\end{align*}
\qed




















\newpage
\section{Locally compact abelian groups}
\ntt Let $\text{TopAb}$ be the category of all Hausdorff topological abelian groups and $\text{LCAb}$ be the category of all locally compact abelian groups.
\prop
Let $A, B\in \text{TopAb}$  and assume that $A\in\text{CGTop}$. Then there is a natural isomorphism of condensed abelian groups
$$
\underline{\text{Hom}}(\underline{A},\underline{B})\cong \underline{\text{Hom}(A,B)}.
$$  




\thm [Eilenberg-Maclane,Breen,Deligne resolution]
For any abelian group $A$, there is a functorial resolution
$$
\cdots\longrightarrow\bigoplus_{j=1}^{n_i}\mathbb{Z}[A^{r_{i,j}}]\longrightarrow\cdots\longrightarrow\mathbb{Z}[A^3]\oplus\mathbb{Z}[A^2]\longrightarrow\mathbb{Z}[A^2]\longrightarrow\mathbb{Z}[A]\longrightarrow A\to 0.
$$
\rem 
Such functorial ensures that it works for abelian group objects in any topos.
\lem\label{double} 
Let $A^{\bullet,\bullet}$ be a double complex and $A^\bullet=\text{Tot}(A^{\bullet,\bullet})$ be its total complex, then there is a spectral sequence
$$
E_1^{p,q}=H^q(A^{\bullet,p})\Longrightarrow H^{p+q}(A^\bullet).
$$
\lem \label{qis}
For a comlpex of abelian groups $M^\bullet\in D(\mathbb{Z})$, let 
$$
0\longrightarrow M^\bullet\longrightarrow A^{\bullet,1}\longrightarrow A^{\bullet,2}\longrightarrow A^{\bullet,3}\longrightarrow\cdots 
$$
be an exact sequence in $D(\mathbb{Z})$, then for the double complex $A^{\bullet,\bullet}$, there is a quasi-isomorphism
$$
M^\bullet\stackrel{\sim}{\rightarrow}\text{Tot}(A^{\bullet,\bullet}).
$$
\cor 
For any condensed abelian groups $A,M$ and an extremally disconnected space $S$, there is a spectral sequence
$$
E_1^{p,q}=\prod_{j=1}^{n_p} H^q(A^{r_{p,j}}\times S,M)\Longrightarrow \underline{\text{Ext}}^{p+q}(A,M)(S),
$$
that is functorial in $A, M$ and $S$.
\pf 
For $A\in \text{Cond(Ab)}$, consider its EMBD resolution
$$
\cdots\longrightarrow\bigoplus_{j=1}^{n_i}\mathbb{Z}[A^{r_{i,j}}]\longrightarrow\cdots\longrightarrow\mathbb{Z}[A^3]\oplus\mathbb{Z}[A^2]\longrightarrow\mathbb{Z}[A^2]\longrightarrow\mathbb{Z}[A]\longrightarrow A\to 0,
$$
then apply $-\otimes \mathbb{Z}[S]$, which is an exact functor since $\mathbb{Z}[S]$ is flat, we get the resolution of $A\otimes \mathbb{Z}[S]$
$$
\cdots\longrightarrow\bigoplus_{j=1}^{n_i}\mathbb{Z}[A^{r_{i,j}}\times S]\cdots\longrightarrow\mathbb{Z}[A^3\times S]\oplus\mathbb{Z}[A^2\times S]\longrightarrow\mathbb{Z}[A^2\times S]\longrightarrow\mathbb{Z}[A\times S]\longrightarrow A\otimes \mathbb{Z}[S]\to 0,
$$
then apply $R\text{Hom}(-,M)$, we get
$$
0\longrightarrow R\text{Hom}(A\otimes \mathbb{Z}[S],M)\longrightarrow R\text{Hom}(\mathbb{Z}[A\times S],M)\longrightarrow R\text{Hom}(\mathbb{Z}[A^2\times S],M)\longrightarrow\cdots,
$$
i.e.
$$
0\longrightarrow R\underline{\text{Hom}}(A,M)(S)\longrightarrow R\Gamma(A\times S,M)\longrightarrow R\Gamma(A^2\times S,M)\longrightarrow \cdots,
$$ 
which is an exact sequence in $D(\mathbb{Z})$. By lemma \ref{double} and lemma \ref{qis}, we have
$$
E_1^{p,q}=H^q(\bigoplus_{j=1}^{n_p} R\Gamma(A^{r_{p,j}}\times S,M))\Longrightarrow H^{p+q}(\text{Tot}(\bigoplus_{j=1}^{n_\bullet} R\Gamma(A^{r_{\bullet,j}}\times S,M)))
$$
and 
$$
R\underline{\text{Hom}}(A,M)(S)\simeq \text{Tot}(\bigoplus_{j=1}^{n_\bullet} R\Gamma(A^{r_{\bullet,j}}\times S,M)),
$$
hence
$$
E_1^{p,q}=\prod_{j=1}^{n_p} H^q(A^{r_{p,j}}\times S,M)\Longrightarrow \underline{\text{Ext}}^{p+q}(A,M)(S).
$$
\qed

\lem \label{exact}
In the category of abelian groups, if the following diagram is exact for each arrow
\begin{equation*}
\xymatrix{
0\ar[r] &M^\bullet\ar[d]\ar[r]&A^{\bullet,1}\ar[d]\ar[r]&A^{\bullet,2}\ar[d]\ar[r]&\cdots \\
0\ar[r] &N^\bullet\ar[r]&B^{\bullet,1}\ar[r]&B^{\bullet,2}\ar[r]&\cdots
},
\end{equation*}
and if for any $j\geq 1$, we have $A^{\bullet,j}\cong B^{\bullet,j}$, then 
$\text{Tot}(A^{\bullet,\bullet})\cong \text{Tot}(B^{\bullet,\bullet})$. Furthermore, by $M^\bullet\cong \text{Tot}(A^{\bullet,\bullet})$ and $N^\bullet\cong \text{Tot}(B^{\bullet,\bullet})$, we can get $M^\bullet\cong N^\bullet.$


\thm 
Assume $I$ is any set, denote the compact condensed abelian group $\prod_I \mathbb{T}$ by $\mathbb{T}^I$.
\begin{itemize}\label{discrete}
	\item [(i)] For any discrete abelian group $M$, we have
	$$
	R\underline{\text{Hom}}(\mathbb{T}^I,M)= M^{\oplus I}[-1],
	$$
	where $M^{\oplus I}[-1]\to R\underline{\text{Hom}}(\mathbb{T}^I,M)$ is induced by 
	$$
	M[-1]=R\underline{\text{Hom}}(\mathbb{Z}[1],M)\longrightarrow R\underline{\text{Hom}}(\mathbb{T},M)\stackrel{p_i^*}{\longrightarrow} R\underline{\text{Hom}}(\mathbb{T}^I,M),
	$$
	where $p_i:\mathbb{T}^I\longrightarrow \mathbb{T}$ is the projection to the $i$-th factor, $i\in I$.
	\label{RHom=0}\item [(ii)]
	$R\underline{\text{Hom}}(\mathbb{T}^I,\mathbb{R})=0.$	 
\end{itemize}
\pf 
\item [(i)]
We first prove the case $I$ is a one element set, i.e.
$$R\underline{\text{Hom}}(\mathbb{T},M)= M[-1].$$
From the exact sequence $0\rightarrow\mathbb{Z}\rightarrow\mathbb{R}\rightarrow\mathbb{T}\rightarrow 0$, we have $\mathbb{R}\to \mathbb{T}\to \mathbb{Z}[1]$, hence 
$$
M[-1]=
R\underline{\text{Hom}}(\mathbb{Z}[1],M)\longrightarrow R\underline{\text{Hom}}(\mathbb{T},M)\longrightarrow
R\underline{\text{Hom}}(\mathbb{R},M).
$$
In order to show $R\underline{\text{Hom}}(\mathbb{T},M)= M[-1]$, it suffices to show $R\underline{\text{Hom}}(\mathbb{R},M)= 0.$

Claim: $R\underline{\text{Hom}}(\mathbb{R},M)= 0.$\\
For 0 and $\mathbb{R}$, we take its EMBD resolution:
\begin{equation*}
\xymatrix
{
\cdots\ar[r]&\bigoplus_{j=1}^{n_i}\mathbb{Z}[\mathbb{R}^{r_{i,j}}]\ar[d]\ar[r] &\cdots  \ar[r]&\mathbb{Z}[\mathbb{R}]\ar[d]\ar[r]&\mathbb{R}\ar[d]\ar[r]&0\\
\cdots\ar[r]&\bigoplus_{j=1}^{n_i}\mathbb{Z}[0^{r_{i,j}}]\ar[r]&\cdots \ar[r]&\mathbb{Z}[0]\ar[r]&0\ar[r] &0,
}
\end{equation*}
apply $R\underline{\text{Hom}}(-,M)(S)$, we get 
\begin{equation*}
\xymatrix
{
0\ar[r]&R\underline{\text{Hom}}(0,M)(S)\ar[d]\ar[r]& R\underline{\text{Hom}}(\mathbb{Z}[0],M)(S)\ar[d]\ar[r]&\cdots\ar[r]& R\underline{\text{Hom}}(\bigoplus_{j=1}^{n_i}\mathbb{Z}[0^{r_{i,j}}],M)(S)\ar[d]\cdots\\
0\ar[r]& R\underline{\text{Hom}}(\mathbb{R},M)(S)\ar[r]& R\underline{\text{Hom}}(\mathbb{Z}[\mathbb{R}],M)(S)\ar[r]&\cdots \ar[r]& R\underline{\text{Hom}}(\bigoplus_{j=1}^{n_i}\mathbb{Z}[\mathbb{R}^{r_{i,j}}],M)(S)\cdots,
}
\end{equation*}
i.e.
\begin{equation*}
\xymatrix
{
	0\ar[r]&0\ar[d]\ar[r]& R\Gamma(S,M)\ar[d]\ar[r]&\cdots\ar[r]& \bigoplus_{j=1}^{n_i}R\Gamma(S,M)\ar[d]\cdots\\
	0\ar[r]& R\underline{\text{Hom}}(\mathbb{R},M)(S)\ar[r]& R\Gamma(\mathbb{R}\times S,M)\ar[r]&\cdots \ar[r]& \bigoplus_{j=1}^{n_i}R\Gamma(\mathbb{R}^{r_{i,j}}\times S,M)\cdots,
}
\end{equation*}
Then by lemma \ref{exact}, in order to show $R\underline{\text{Hom}}(\mathbb{R},M)=0$, it suffices to show 
$$
R\Gamma(S,M)=R\Gamma(S\times \mathbb{R}^r,M).
$$
We know $S\times \mathbb{R}^r=\underrightarrow{\text{lim}}\  S\times [-N,N]^r$, then
\begin{align*}
R\Gamma(S\times \mathbb{R}^r,M)
&=R\Gamma(\underrightarrow{\text{lim}}\  S\times [-N,N]^r,M)\\
&=\underleftarrow{\text{lim}}\ R\Gamma(S\times [-N,N]^r,M)\\
&=\underleftarrow{\text{lim}}\ R\Gamma(S,M)\\
&=R\Gamma(S,M).
\end{align*}
Here, $\underleftarrow{\text{lim}}\ R\Gamma(S\times [-N,N]^r,M)=\underleftarrow{\text{lim}}\ R\Gamma(S,M)$ comes from the fact that for constant sheaf, its sheaf cohomology is homotopy-invariant.
~\\
Secondly, assume $I$ is a finite set, then
$$
R\underline{\text{Hom}}(\mathbb{T}^I,M)=R\underline{\text{Hom}}(\mathbb{T}^{\oplus I},M)=\prod_I R\underline{\text{Hom}}(\mathbb{T},M)=\prod_I M[-1]=M^{\oplus I}[-1].
$$
Finally, assume $I$ is any set. Then we can write $\mathbb{T}^I$ as $$\mathbb{T}^I=\underset{J\subset I, J \text{ finite}}{\underleftarrow{\text{lim}}}\  \mathbb{T}^J.$$
For any finite set $J$, we have
\begin{equation*}
\xymatrix
{
	0\ar[r]&R\underline{\text{Hom}}(\mathbb{T}^J,M)(S)\ar[d]\ar[r]& R\Gamma(\mathbb{T}^J\times S,M)\ar[d]\ar[r]&\cdots\ar[r]& \bigoplus_{j=1}^{n_i}R\Gamma((\mathbb{T}^J)^{r_{i,j}}\times S,M)\ar[d]\cdots\\
	0\ar[r]& R\underline{\text{Hom}}(\mathbb{T}^I,M)(S)\ar[r]& R\Gamma(\mathbb{T}^I\times S,M)\ar[r]&\cdots \ar[r]& \bigoplus_{j=1}^{n_i}R\Gamma((\mathbb{T}^I)^{r_{i,j}}\times S,M)\cdots,
}
\end{equation*}
apply the exact functor $\underset{J\subset I}{\underrightarrow{\text{lim}}}$ to the first arrow, we get
\begin{equation*}
\xymatrix
{
	0\ar[r]&\underset{J\subset I}{\underrightarrow{\text{lim}}} R\underline{\text{Hom}}(\mathbb{T}^J,M)(S)\ar[d]\ar[r]& \underset{J\subset I}{\underrightarrow{\text{lim}}} R\Gamma(\mathbb{T}^J\times S,M)\ar[d]\ar[r]&\cdots\ar[r]& \bigoplus_{j=1}^{n_i} \underset{J\subset I}{\underrightarrow{\text{lim}}} R\Gamma((\mathbb{T}^J)^{r_{i,j}}\times S,M)\ar[d]\cdots\\
	0\ar[r]& R\underline{\text{Hom}}(\mathbb{T}^I,M)(S)\ar[r]& R\Gamma(\mathbb{T}^I\times S,M)\ar[r]&\cdots \ar[r]& \bigoplus_{j=1}^{n_i}R\Gamma((\mathbb{T}^I)^{r_{i,j}}\times S,M)\cdots,
}
\end{equation*}
In order to show
$$
\underset{J\subset I}{\underrightarrow{\text{lim}}} R\underline{\text{Hom}}(\mathbb{T}^J,M)(S)\cong R\underline{\text{Hom}}(\mathbb{T}^I,M)(S),
$$ 
it suffices to show 
$$
\underset{J\subset I}{\underrightarrow{\text{lim}}} R\Gamma((\mathbb{T}^J)^{r_{i,j}}\times S,M)\cong R\Gamma((\mathbb{T}^I)^{r_{i,j}}\times S,M).
$$ 
This is true, because $\underset{J\subset I}{\underleftarrow{\text{lim}}}\ (\mathbb{T}^J)^{r_{i,j}}\times S\cong  (\mathbb{T}^I)^{r_{i,j}}\times S.$
Therefore, 
\begin{align*}
R\underline{\text{Hom}}(\mathbb{T}^I,M)
&\cong \underset{J\subset I}{\underrightarrow{\text{lim}}} R\underline{\text{Hom}}(\mathbb{T}^J,M)\\
&\cong \underset{J\subset I}{\underrightarrow{\text{lim}}} M^{\oplus J}[-1]\\
&\cong M^{\oplus I}[-1].
\end{align*}
\qed

\cor 
$R\underline{\text{Hom}}(\mathbb{R},\mathbb{R})\cong \mathbb{R}.$
\pf 
From the exact sequence $0\to \mathbb{Z}\to \mathbb{R}\to \mathbb{T}\to 0$, we have 
$$R\underline{\text{Hom}}(\mathbb{T},\mathbb{R})\rightarrow R\underline{\text{Hom}}(\mathbb{R},\mathbb{R})\rightarrow R\underline{\text{Hom}}(\mathbb{Z},\mathbb{R}).$$
By Theorem \ref{RHom=0}, we know $R\underline{\text{Hom}}(\mathbb{T},\mathbb{R})=0$, hence $R\underline{\text{Hom}}(\mathbb{R},\mathbb{R})\cong R\underline{\text{Hom}}(\mathbb{Z},\mathbb{R})\cong \mathbb{R}.$
\qed

\cor 
For any locally compact abelian groups $A$ and $B$, $R\underline{\text{Hom}}(A,B)$ is centered at 0 and 1, i.e. $\underline{\text{Ext}}^i(A,B)=0,\ \forall i\geq 2.$
\pf 
By the structure theorem of locally compact abelian groups, it suffices to prove for $A$ and $B$ being compact groups and discrete groups.
\begin{itemize}
	\item [(i)] $A$ is a discrete group.\\
	Claim: There is an exact sequence: $0\to \oplus_I\mathbb{Z} \to \oplus_J\mathbb{Z}\to A\to 0.$\\
	This is because we can construct a surjective homomorphism $\oplus_A \mathbb{Z}\to A$, and take its kernel, and we know the submodule of a free $\mathbb{Z}$-module is free, hence $\text{ker}(\oplus_A \mathbb{Z}\to A)=\oplus_I \mathbb{Z}$, for some $I$. Thereby, $0\to \oplus_I\mathbb{Z} \to \oplus_A\mathbb{Z}\to A\to 0$ is exact.\\
	By the short exact sequence $0\to \oplus_I\mathbb{Z} \to \oplus_J\mathbb{Z}\to A\to 0$, we can get a long exact sequence:
	\begin{align*}
	0&\longrightarrow \underline{\text{Hom}}(A,B)\longrightarrow \underline{\text{Hom}}(\oplus_J\mathbb{Z},B)\longrightarrow \underline{\text{Hom}}(\oplus_I\mathbb{Z},B)\\
	&\longrightarrow \underline{\text{Ext}}^1(A,B)\longrightarrow \underline{\text{Ext}}^1(\oplus_J\mathbb{Z},B)\longrightarrow \underline{\text{Ext}}^1(\oplus_I\mathbb{Z},B)\\
	&\longrightarrow \underline{\text{Ext}}^2(A,B)\longrightarrow \cdots.
	\end{align*}
	Because $\oplus_I\mathbb{Z}\in \text{Cond(Ab)}$ is projective, we have $\underline{\text{Ext}}^i(\oplus_I\mathbb{Z},B)=0,\ \forall i\geq 1.$ Hence $\underline{\text{Ext}}^i(A,B)=0,\ \forall i\geq 2.$
	\item [(ii)] $A$ is a compact group.\\
	By Pontrgagin duality, there is a short exact sequence
	$$0\rightarrow A\rightarrow \mathbb{T}^I\rightarrow \mathbb{T}^J\rightarrow 0,$$ and it can induce a long exact sequence
	\begin{align*}
	0&\longrightarrow \underline{\text{Hom}}(\mathbb{T}^J,B)\longrightarrow \underline{\text{Hom}}(\mathbb{T}^I,B)\longrightarrow \underline{\text{Hom}}(A,B)\\
	&\longrightarrow \underline{\text{Ext}}^1(\mathbb{T}^J,B)\longrightarrow \underline{\text{Ext}}^1(\mathbb{T}^I,B)\longrightarrow \underline{\text{Ext}}^1(A,B)\\
	&\longrightarrow \underline{\text{Ext}}^2(\mathbb{T}^J,B)\longrightarrow \underline{\text{Ext}}^2(\mathbb{T}^I,B)\longrightarrow
	\underline{\text{Ext}}^2(A,B)\\
	&\longrightarrow\cdots
	\end{align*}
	In order to show $\underline{\text{Ext}}^i(A,B)=0,\ \forall i\geq 2$, it suffices to show 
	$$\underline{\text{Ext}}^i(\mathbb{T}^I,B)=0,\ \forall i\geq 2,\  \forall I.$$
	\begin{itemize}
		\item [(a)] $B$ is a discrete group.\\
		In this case, we have $R\underline{\text{Hom}}(\mathbb{T}^I,B)=B^{\oplus I}[-1]$, which is centered at 1, hence $\underline{\text{Ext}}^i(\mathbb{T}^I,B)=0,\ \forall i\geq 2,\  \forall I.$
		\item [(b)] $B$ is a compact group.\\
		In this case, we have a short exact sequence $0\rightarrow B\rightarrow \mathbb{T}^{I^\prime}\rightarrow \mathbb{T}^{J^\prime}\rightarrow 0$, and it induces a long exact sequence:
		\begin{align*}
		0&\longrightarrow \underline{\text{Hom}}(\mathbb{T}^I,B)\longrightarrow \underline{\text{Hom}}(\mathbb{T}^I,\mathbb{T}^{I^\prime})\longrightarrow \underline{\text{Hom}}(\mathbb{T}^I,\mathbb{T}^{J^\prime})\\
		&\longrightarrow \underline{\text{Ext}}^1(\mathbb{T}^I,B)\longrightarrow \underline{\text{Ext}}^1(\mathbb{T}^I,\mathbb{T}^{I^\prime})\longrightarrow \underline{\text{Ext}}^1(\mathbb{T}^I,\mathbb{T}^{J^\prime})\\
		&\longrightarrow \underline{\text{Ext}}^2(\mathbb{T}^I,B)\longrightarrow \cdots.
		\end{align*}
		 Now, we compute $\underline{\text{Ext}}^i(\mathbb{T}^I,\mathbb{T})$. For the short exact sequence $0\to \mathbb{Z}\to \mathbb{R}\to \mathbb{T}\to 0$, we have a long exact sequence:
		 \begin{align*}
		 0&\longrightarrow \underline{\text{Hom}}(\mathbb{T}^I,\mathbb{Z})\longrightarrow \underline{\text{Hom}}(\mathbb{T}^I,\mathbb{R})\longrightarrow \underline{\text{Hom}}(\mathbb{T}^I,\mathbb{T})\\
		 &\longrightarrow \underline{\text{Ext}}^1(\mathbb{T}^I,\mathbb{Z})\longrightarrow \underline{\text{Ext}}^1(\mathbb{T}^I,\mathbb{R})\longrightarrow \underline{\text{Ext}}^1(\mathbb{T}^I,\mathbb{T})\\
		 &\longrightarrow \underline{\text{Ext}}^2(\mathbb{T}^I,\mathbb{Z})\longrightarrow \cdots.
		 \end{align*}
		 Since $R\underline{\text{Hom}}(\mathbb{T}^I,\mathbb{R})=0$ and $R\underline{\text{Hom}}(\mathbb{T}^I,\mathbb{Z})=\mathbb{Z}^{\oplus I}[-1]$, we have $\underline{\text{Ext}}^i(\mathbb{T}^I,\mathbb{T})=0,\ \forall i\geq 1$, hence $\underline{\text{Ext}}^i(\mathbb{T}^I,\mathbb{T}^J)=0,\ \forall i\geq 1,\ \forall J$. Thus $\underline{\text{Ext}}^i(\mathbb{T}^I,B)=0,\ \forall i\geq 2.$
	\end{itemize}
\end{itemize}
\qed

\newpage
\section{Solid Abelian Groups}
\dfn 
For $S\in\text{ProFin}$, write $S=\underleftarrow{\text{lim}}\ 
 S_i$, where $S_i\in \text{Fin}$, we define the solid free abelian group
 $$
 \mathbb{Z}[S]^{\blacksquare}:=\underleftarrow{\text{lim}}\ \mathbb{Z}[S_i].
 $$
We call $ \mathbb{Z}[S]^{\blacksquare}$ the solidification of $\mathbb{Z}[S]$.

\rem~\\
$
 \mathbb{Z}[S]^{\blacksquare}=\underleftarrow{\text{lim}}\ \mathbb{Z}[S_i]
 =\underleftarrow{\text{lim}}\ \underline{\text{Hom}}(C(S_i,\mathbb{Z}),\mathbb{Z})
 =\underline{\text{Hom}}(\underrightarrow{\text{lim}}\ C(S_i,\mathbb{Z}),\mathbb{Z})
 =\underline{\text{Hom}}(C(S,\mathbb{Z}),\mathbb{Z}).
$

\prop 
For $S\in\text{ProFin}$, there exists some set $I$, s.t. $C(S,\mathbb{Z})\cong \mathbb{Z}^{\oplus I}$, i.e. $C(S,\mathbb{Z})$ is a free abelian group.
\rem
\begin{itemize}
	\item [(i)]From the above proposition, we have
	$$\mathbb{Z}[S]^{\blacksquare}=\underline{\text{Hom}}(C(S,\mathbb{Z}),\mathbb{Z})=\underline{\text{Hom}}(\mathbb{Z}^{\oplus I},\mathbb{Z})=\mathbb{Z}^I.$$
	\item [(ii)]
\end{itemize}

\dfn 
A condensed abelian group $X\in\text{Cond(Ab)}$ is solid, if for any $S\in\text{ProFin}$, one has
$$
\text{Hom}(\mathbb{Z}[S],X)\cong \text{Hom}(\mathbb{Z}[S]^{\blacksquare},X).
$$
A complex of condensed abelian groups $C\in D(\text{Cond(Ab)})$ is solid, if for any $S\in\text{ProFin}$, one has
$$
R\text{Hom}(\mathbb{Z}[S],C)\cong R\text{Hom}(\mathbb{Z}[S]^{\blacksquare},C).
$$
Now, we need to check $\mathbb{Z}[S]^{\blacksquare}$ is indeed a solid condensed abelian group.

\prop 
For $S, T\in\text{ProFin}$, we have
$$
R\text{Hom}(\mathbb{Z}[S],\mathbb{Z}[T]^{\blacksquare})\cong R\text{Hom}(\mathbb{Z}[S]^{\blacksquare},\mathbb{Z}[T]^{\blacksquare}).
$$
\pf
Assume $\mathbb{Z}[S]^{\blacksquare}=\mathbb{Z}^I$ and $\mathbb{Z}[T]^{\blacksquare}=\mathbb{Z}^J$ for some sets $I$ and $J$. Since the functors $R\text{Hom}(\mathbb{Z}[S],-)$ and $R\text{Hom}(\mathbb{Z}[S]^{\blacksquare},-)$ commute with products, it suffices to show 
$$
R\text{Hom}(\mathbb{Z}[S],\mathbb{Z})\cong R\text{Hom}(\mathbb{Z}[S]^{\blacksquare},\mathbb{Z})
$$
The left hand side is $R\text{Hom}(\mathbb{Z}[S],\mathbb{Z})\cong R\Gamma(S,\mathbb{Z})=C(S,\mathbb{Z})=\mathbb{Z}^{\oplus I}.$\\
Now, consider the short exact sequence $0\rightarrow \mathbb{R}^I\rightarrow \mathbb{Z}^I\rightarrow \mathbb{T}^I\rightarrow 0.$ From theorem \ref{discrete}, We know $$R\text{Hom}(\mathbb{T}^I,\mathbb{Z})=\mathbb{Z}^{\oplus I}[-1].$$
 And by the adjoint relation, we have
 $$
 R\text{Hom}(\mathbb{R}^I,\mathbb{Z})\cong R\text{Hom}_{\mathbb{R}}(\mathbb{R}^I,R\underline{\text{Hom}}(\mathbb{R},\mathbb{Z}))=0.
 $$
 Hence, $R\text{Hom}(\mathbb{Z}[S]^{\blacksquare},\mathbb{Z})\cong R\text{Hom}(\mathbb{Z}^I,\mathbb{Z})\cong\mathbb{Z}^{\oplus I}.$ And this finishes our proof.
\qed
\\
\lem 
Let $\mathcal{A}$ be a cocomplete abelian category, and $\mathcal{A}_0\subseteq \mathcal{A}$ be the full subcategory of compact projective generators. Assume $F:\mathcal{A}_0\rightarrow \mathcal{A}$ is an additive functor with a natural transformation $\mathrm{id}_{\mathcal{A}_0}\implies F$, satisfying the following property:
~\\

For any $X\in\mathcal{A}_0$, any $Y, Z\in\mathcal{A}$ which can be written as direct sums of objects in the image of $F$, i.e. $Y=\bigoplus_{i\in I} F(X_i) $ and $Z=\bigoplus_{j\in J} F(X_j)$, and for any map $f:Y\rightarrow Z$ with kernel $K\in\mathcal{A}$, the map
$$
R\text{Hom}(F(X),K)\rightarrow R\text{Hom}(X,K)
$$
is an isomorphism.
~\\

Let 
$$\mathcal{A}_F=\left\{Y\in\mathcal{A}\mid  
\text{Hom}(F(X),Y)\cong \text{Hom}(X,Y), \forall X\in\mathcal{A}_0
\right\}\subseteq \mathcal{A}$$
and 
$$
D_F(\mathcal{A})=\left\{C\in D(\mathcal{A})\mid  
R\text{Hom}(F(X),C)\cong R\text{Hom}(X,C), \forall X\in\mathcal{A}_0
\right\}\subseteq D(\mathcal{A})
$$
Then:
\begin{itemize}
\item [(i)]
	\begin{itemize}
		\item [-]	$\mathcal{A}_F \subseteq \mathcal{A}$ is an abelian subcategory stable under limits, colimits and extensions.
		\item [-]The objects $F(X), X\in \mathcal{A}_0$ are compact projective generators.
		\item [-]	The inclusion $\mathcal{A}_F\hookrightarrow \mathcal{A}$ admits a left adjoint $L:\mathcal{A}\rightarrow \mathcal{A}_F$, which is the unique colimit-preserving extension of $F:\mathcal{A}_0\rightarrow\mathcal{A}_F.$
	\end{itemize}
\item [(ii)] 
	\begin{itemize}
		\item [-]	The functor $D(\mathcal{A}_F)\rightarrow D(\mathcal{A})$ is fully faithful and $D(\mathcal{A}_F)\cong D_F(\mathcal{A}).$
		\item [-]$C\in D(\mathcal{A})$ lies in $D_F(\mathcal{A})$ iff $H^i(C)\in \mathcal{A}_F.$
		\item [-]	The above functor $F$ has a left derived functor, which is the left adjoint of $D_F(\mathcal{A})\hookrightarrow D(\mathcal{A}).$
	\end{itemize}
\end{itemize}

\lem We take the above lemma's notation.
\begin{itemize}
	\item [(i)] For any $C$ with the form $\bigoplus_{i\in I} F(X_i),  X_i\in\mathcal{A}_0$, one has
	$$
	R\text{Hom}(F(X),C)\cong R\text{Hom}(X,C),\ \forall X\in\mathcal{A}_0.
	$$
	\item [(ii)]For any $C$ with the form $\text{ker}(\bigoplus_{i\in I} F(X_i)\rightarrow \bigoplus_{j\in J} F(Y_j))$, $X_i, Y_j\in \mathcal{A}_0$, one has
	$$
		R\text{Hom}(F(X),C)\cong R\text{Hom}(X,C),\ \forall X\in\mathcal{A}_0.
	$$
	\item [(iii)]
	For any $C$ with the form $\text{coker}(\bigoplus_{i\in I} F(X_i)\rightarrow \bigoplus_{j\in J} F(Y_j))$, $X_i, Y_j\in \mathcal{A}_0$, one has
	$$
	R\text{Hom}(F(X),C)\cong R\text{Hom}(X,C),\ \forall X\in\mathcal{A}_0.
	$$
	\item [(iv)] For any right bounded complex $C$ with each term $C_i$ having the form $\bigoplus_{j\in I_i} F(X_{i_j})$, one has
	$$
	R\text{Hom}(F(X),C)\cong R\text{Hom}(X,C),\ \forall X\in\mathcal{A}_0.
	$$
	Then (iv)$\implies$ (iii)$\iff$ (ii) $\implies$ (i).
\end{itemize}
\pf
(ii)$\implies$ (i). Just take $J=\emptyset$, which is exactly (i).\\
(ii)$\iff$ (iii). For any $f: Y\rightarrow Z$, with $Y=\bigoplus_{i\in I} F(X_i)$ and $Z=\bigoplus_{j\in J} F(Y_j)$, applying functors $R\text{Hom}(X,-)$ and $R\text{Hom}(F(X),-)$ to the exact sequence:
$$
0\rightarrow \text{ker}(f)\rightarrow Y\rightarrow Z\rightarrow \text{coker}(f)\rightarrow 0,
$$
one get 
\begin{equation*}
\xymatrix{
	R\text{Hom}(F(X),\text{ker}(f)) \ar[r]\ar[d] &R\text{Hom}(F(X),Y)\ar[d]^\cong\ar[r]&R\text{Hom}(F(X),Z)\ar[d]^\cong\ar[r]&R\text{Hom}(F(X),\text{coker}(f))\ar[d] \\
	R\text{Hom}(X,\text{ker}(f))\ar[r] &R\text{Hom}(X,Y)\ar[r]&R\text{Hom}(X,Z)\ar[r]&R\text{Hom}(X,\text{coker}(f))
}
\end{equation*}
By five lemma, we can show 
$$
R\text{Hom}(F(X),\text{ker}(f))\cong R\text{Hom}(X,\text{ker}(f))
$$
$\iff$
$$
R\text{Hom}(F(X),\text{coker}(f))\cong R\text{Hom}(X,\text{coker}(f)).
$$
Hence, (ii)$\iff$(iii).\\
(iv)$\implies$ (ii). For any $f: Y\rightarrow Z$, with $Y=\bigoplus_{i\in I} F(X_i)$ and $Z=\bigoplus_{j\in J} F(Y_j)$. Denote $K=\text{ker}(f)$. Take the resolution of $K$:
$$
\cdots\rightarrow B_1\rightarrow B_0\rightarrow K\rightarrow 0,
$$
where each $B_i\in\mathcal{A}_0$. Now, take $C=[0\rightarrow Y\rightarrow Z\rightarrow 0]$, by assumption, we have
$$
R\text{Hom}(F(B_{\bullet}),C)\cong R\text{Hom}(B_{\bullet},C).
$$
Hence,
$$
\begin{tikzcd}
B_{\bullet} \arrow[d] \arrow[r] & F(B_{\bullet}) \arrow[ld, "\exists !", dotted] \\
K                               &                                               
\end{tikzcd}
$$
That is, $K\cong B_\bullet$ is the retract of $F(B_\bullet).$
Thus, 
$$
R\text{Hom}(X,K)\cong R\text{Hom}(X,F(B_\bullet))\cong R\text{Hom}(F(X),F(B_\bullet))\cong R\text{Hom}(F(X),K).
$$
\qed

\thm 

\begin{itemize}
	\item [(i)]
	\begin{itemize}
		\item [-]
		The category $\text{Solid}\subset\text{Cond(Ab)}$ of solid abelian groups is an abelian subcategory stable under limits, colimits and extensions.
	\end{itemize}
	\item [(ii)]
\end{itemize}





\dfn 
\begin{itemize}
	\item [(i)]For $M, N\in\text{Solid}$, define $M\otimes^\blacksquare N:=(M\otimes N)^\blacksquare.$
	\item [(ii)]For $C, D\in D(\text{Solid})$, define $C\otimes^{L\blacksquare} D:=(C\otimes^L D)^{L\blacksquare}.$
\end{itemize}

\thm 
\begin{itemize}
	\item [(i)] The solidification functor $\text{Cond(Ab)}\rightarrow \text{Solid};\ M\mapsto M^\blacksquare$ is symmetric monoidal, i.e.
	$$
	(M\otimes N)^\blacksquare\cong M^\blacksquare \otimes^\blacksquare N^\blacksquare.
	$$
	\item [(ii)]
	 The solidification functor $D(\text{Cond(Ab)})\rightarrow D(\text{Solid});\ C\mapsto C^{L\blacksquare}$ is symmetric monoidal, i.e.
	$$
	(C\otimes^L D)^{L\blacksquare}\cong C^{L\blacksquare} \otimes^{L\blacksquare} D^{L\blacksquare}.
	$$
	\item [(iii)]$\otimes^{L\blacksquare}$ is the left derived functor of $\otimes^{\blacksquare}$.
\end{itemize}

\pf
\begin{itemize}
	\item [(i)] By definition, we need to show:
	$$
	(M\otimes N)^\blacksquare\stackrel{\sim}{\longrightarrow} (M^\blacksquare\otimes N^\blacksquare)^\blacksquare.
	$$
	This can be written as the composition:
	$$
	(M\otimes N)^\blacksquare \longrightarrow (M^\blacksquare\otimes N)^\blacksquare\longrightarrow (M^\blacksquare\otimes N^\blacksquare)^\blacksquare.
	$$
	Hence, it is enough to prove 
	$$
		(M\otimes N)^\blacksquare \stackrel{\sim}{\longrightarrow} (M^\blacksquare\otimes N)^\blacksquare.
	$$
	(With this isomorphism, we can also show that the second map is an isomorphism). Since the tensor functor and the solidification functor commute with colimits, then we can assume $M=\mathbb{Z}[S]$ and $N=\mathbb{Z}[T].$\\
	It reduces to show:
	$$
	\mathbb{Z}[S\times T]^\blacksquare\stackrel{\sim}{\longrightarrow}
	(\mathbb{Z}[S]^\blacksquare\otimes \mathbb{Z}[T])^\blacksquare.
	$$
	Equivalently, for any $A\in\text{Solid}$,
	$$
	\underline{\text{Hom}}((\mathbb{Z}[S]^\blacksquare\otimes \mathbb{Z}[T])^\blacksquare, A)\cong 	\underline{\text{Hom}}(\mathbb{Z}[S\times T]^\blacksquare, A).
	$$
	Since $A$ is solid, we have:
	$$
	\underline{\text{Hom}}((\mathbb{Z}[S]^\blacksquare\otimes \mathbb{Z}[T])^\blacksquare, A)\cong \underline{\text{Hom}}(\mathbb{Z}[S]^\blacksquare\otimes \mathbb{Z}[T], A)
	$$
	and 
	$$
	\underline{\text{Hom}}(\mathbb{Z}[S\times T]^\blacksquare, A)\cong\underline{\text{Hom}}(\mathbb{Z}[S\times T], A).
	$$
	By computation:
	\begin{align*}
	\underline{\text{Hom}}(\mathbb{Z}[S]^\blacksquare\otimes \mathbb{Z}[T], A)
	&\cong 	\underline{\text{Hom}}(\mathbb{Z}[S]^\blacksquare,\underline{\text{Hom}}(\mathbb{Z}[T], A))\\
	&\cong 	\underline{\text{Hom}}(\mathbb{Z}[S],\underline{\text{Hom}}(\mathbb{Z}[T], A))\\
	&\cong \underline{\text{Hom}}(\mathbb{Z}[S]\otimes\mathbb{Z}[T],A)\\
	&\cong \underline{\text{Hom}}(\mathbb{Z}[S\times T], A).
	\end{align*}
	Thus, $	\underline{\text{Hom}}((\mathbb{Z}[S]^\blacksquare\otimes \mathbb{Z}[T])^\blacksquare, A)\cong 	\underline{\text{Hom}}(\mathbb{Z}[S\times T]^\blacksquare, A).$
	\item [(ii)] Similar to the proof of (i).
	\item [(iii)]
\end{itemize}

\rem 
In $\text{Solid}$, $\otimes^\blacksquare$ is the left adjoint of $\underline{\text{Hom}}$:
$$
\text{Hom}(M\otimes^\blacksquare N,P)\cong
\text{Hom}((M\otimes N)^\blacksquare,P)\cong 
\text{Hom}(M\otimes N,P)\cong
\text{Hom}(M,\underline{\text{Hom}}(N,P)).
$$

\prop 
\begin{itemize}
	\item [(i)] If $X\in \text{CHaus}$, then $\mathbb{Z}[X]^{L\blacksquare}=R\underline{\text{Hom}}(R\Gamma(X,\mathbb{Z}),\mathbb{Z}).$ \\In particular, if $X\in\text{ProFin}\subseteq \text{CHaus}$, then $\mathbb{Z}[X]^{L\blacksquare}=\mathbb{Z}[X]^{\blacksquare}.$
	\item [(ii)] If $X$ is a CW space, then $\mathbb{Z}[X]^{L\blacksquare}=C_\bullet(X).$ \\
	This shows that the derived solidification of a condensed abelian group can sit in all nonnegative homological degrees.
\end{itemize}












\prop 
\begin{itemize}
	\item [(i)] $\mathbb{R}^{L\blacksquare}=0.$
	\item [(ii)] $\mathbb{Z}^I\otimes^{L\blacksquare} \mathbb{Z}^J=\mathbb{Z}^{I\times J}.$
	\item [(iii)] $\mathbb{Z}_p\otimes^{L\blacksquare}\mathbb{Z}_p=\mathbb{Z}_p.$
	\item [(iv)]$\mathbb{Z}_p\otimes^{L\blacksquare}\mathbb{Z}_l=\mathbb{Z}_p.\ (p\neq l)$
\end{itemize}
\pf 
\begin{itemize}
	\item [(i)]
	By Yoneda's lemma, it suffices to show: for any $
	C\in D(\text{Solid})$, one has 
	$$
	R\text{Hom}(\mathbb{R}^{L\blacksquare},C)=
	R\text{Hom}(\mathbb{R},C)=0.
	$$
	
	Since $C=\underleftarrow{\text{lim}}\ C_{\geq n}$, and $R\text{Hom}(\mathbb{R},-)$ commutes with limits, it reduces to the case $C$ is a right bounded complex. And for a right bounded complex $C$, one has $C=\underleftarrow{\text{lim}}\ C_{\leq n}$, it reduces to the case $C$ is a bounded complex.\\
	 Hence it suffices to show: for any $X\in \text{Solid}$, one has $R\text{Hom}(\mathbb{R},X)=0.$ \\
	We know for any object $X\in \text{Solid}$, we can write $X$ as the colimit of objects of the form $\bigoplus_{j\in J}\mathbb{Z}^{I_j}.$ And we know taking all colimits is equivalent to taking all cokernels and all filtered colimits.\\
	Since $\mathbb{R}$ is pseudo-coherent, we get 
	$$
	R\text{Hom}(\mathbb{R},\underrightarrow{\text{lim}}\  \bigoplus_{i\in J_j}\mathbb{Z}^{I_{i,j}})=\underrightarrow{\text{lim}}\ 
		R\text{Hom}(\mathbb{R},\bigoplus_{i\in J_j}\mathbb{Z}^{I_{i,j}})=\underrightarrow{\text{lim}}\ \bigoplus_{i\in J_j}
		R\text{Hom}(\mathbb{R},\mathbb{Z}^{I_{i,j}})=0.
	$$
	Let $f:X\rightarrow Y$, $X=\bigoplus_{i\in I}\mathbb{Z}^{I_i}$ and $Y=\bigoplus_{j\in J}\mathbb{Z}^{I_j}$, then from $R\text{Hom}(\mathbb{R},X)=0$ and $R\text{Hom}(\mathbb{R},Y)=0$, we know $R\text{Hom}(\mathbb{R},\text{coker}(f))=0.$\\
	Thus, we finish our proof.
	
	\item [(ii)] Assume $\mathbb{Z}^I=\mathbb{Z}[S]^\blacksquare=\underline{\text{Hom}}(C(S,\mathbb{Z}),\mathbb{Z}),\  \mathbb{Z}^J=\mathbb{Z}[T]^\blacksquare=\underline{\text{Hom}}(C(T,\mathbb{Z}),\mathbb{Z})$, for some $S,\ T\in \text{ProFin}.$ 
	Then
	\begin{align*}
	\mathbb{Z}[S\times T]^{\blacksquare}
	&=\underline{\text{Hom}}(C(S\times T,\mathbb{Z}),\mathbb{Z})\\
	&=\underline{\text{Hom}}(C(S,\mathbb{Z})\otimes C(T,\mathbb{Z}),\mathbb{Z})\\
	&=\underline{\text{Hom}}(C(S,\mathbb{Z}),\underline{\text{Hom}}( C(T,\mathbb{Z}),\mathbb{Z}))\\
	&=\underline{\text{Hom}}(C(S,\mathbb{Z}),\mathbb{Z}^J)\\
	&=\underline{\text{Hom}}(C(S,\mathbb{Z}),\mathbb{Z})^J\\
	&=\mathbb{Z}^{I\times J}.
	\end{align*}
	Thus, we have
	$$
	\mathbb{Z}^I\otimes^{L\blacksquare} \mathbb{Z}^J=\mathbb{Z}[S]^\blacksquare\otimes^{L\blacksquare}\mathbb{Z}[T]^\blacksquare=(\mathbb{Z}[S]\otimes^{L}\mathbb{Z}[T])^{L\blacksquare}=\mathbb{Z}[S\times T]^{L\blacksquare}=\mathbb{Z}^{I\times J}.
	$$
	\item [(iii)]
	\item [(iv)]
\end{itemize}
\qed






	
\end{document}